\documentclass[a4paper,12pt]{article}
\usepackage{amsmath,amssymb,amsfonts,amsthm}
\usepackage{tikz}
\usepackage [utf8] {inputenc}
\usepackage [T2A] {fontenc} 
\usepackage[russian]{babel}
\usepackage{cmap}
\usepackage{listings}
\usepackage{color}

\definecolor{mygreen}{rgb}{0,0.6,0}
\definecolor{mygray}{rgb}{0.5,0.5,0.5}
\definecolor{mymauve}{rgb}{0.58,0,0.82} 

\lstdefinestyle{customc}{
	belowcaptionskip=1\baselineskip,
	breaklines=true,
	frame=L,
	xleftmargin=\parindent,
	language=Python,
	showstringspaces=false,
	basicstyle=\footnotesize\ttfamily,
	keywordstyle=\bfseries\color{green!40!black},
	commentstyle=\itshape\color{purple!40!black},
	identifierstyle=\color{blue},
	stringstyle=\color{orange},
}


\lstset{escapechar=@,style=customc}

% Так ссылки в PDF будут активны
\usepackage[unicode]{hyperref}

% вы сможете вставлять картинки командой \includegraphics[width=0.7\textwidth]{ИМЯ ФАЙЛА}
% получается подключать, как минимум, файлы .pdf, .jpg, .png.
\usepackage{graphicx}
% Если вы хотите явно указать поля:
\usepackage[margin=1in]{geometry}
% Или если вы хотите задать поля менее явно (чем больше DIV, тем больше места под текст):
% \usepackage[DIV=10]{typearea}

\usepackage{fancyhdr}

\newcommand{\bbR}{\mathbb R}%теперь вместо длинной команды \mathbb R (множество вещественных чисел) можно писать короткую запись \bbR. Вместо \bbR вы можете вписать любую строчку букв, которая начинается с '\'.
\newcommand{\eps}{\varepsilon}
\newcommand{\bbN}{\mathbb N}
\newcommand{\dif}{\mathrm{d}}

\pagestyle{fancy}
\makeatletter % сделать "@" "буквой", а не "спецсимволом" - можно использовать "служебные" команды, содержащие @ в названии
\fancyhead[L]{\footnotesize Информатика}%Это будет написано вверху страницы слева
\fancyhead[R]{\footnotesize МФТИ}
\fancyfoot[L]{\footnotesize \@author}%имя автора будет написано внизу страницы слева
\fancyfoot[R]{\thepage}%номер страницы —- внизу справа
\fancyfoot[C]{}%по центру внизу страницы пусто

\renewcommand{\maketitle}{%
	\noindent{\bfseries\scshape\large\@title\ \mdseries\upshape}\par
	\noindent {\large\itshape\@author}
	\vskip 2ex}
\makeatother
\def\dd#1#2{\frac{\partial#1}{\partial#2}}
\def\ddd#1#2#3{\frac{\partial^2#1}{\partial#2\partial#3}}

\title{Графы. I} 
\author{Тимофей Хирьянов} 
\date{27 февраля 2016 г.}

\begin{document}
	\maketitle
	\section{Способы представления графа в памяти}
		Существует три основных способа представления графа в памяти:
		\begin{enumerate}
			\item Список ребер
			\item Матрица смежности
			\item Список связности
		\end{enumerate}
		На практике граф обычно представляется в виде списка ребер. Ниже вы видите представления одного и того же графа разными способами.
		
		\subsection{Список ребер}
			\lstinputlisting[language=Python]{listofedges.py}
			Считывание списка ребер в список тривиально, поэтому опустим его.
			
		\subsection{Матрица смежности}
			\lstinputlisting[language=Python]{matrix.py}
			Считывание списка ребер в матрицу смежности:
			\lstinputlisting[language=Python]{readmatrix.py}
			
		\subsection{Список связности}
			\lstinputlisting[language=Python]{list.py}
			\newpage
			Перейдем от списка связности к списку ребер:
			\lstinputlisting[language=Python]{createlistofedges.py}
			Считывание списка ребер в список связности:
			\lstinputlisting[language=Python]{readlist.py}
	
\end{document}
