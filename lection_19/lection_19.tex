\documentclass[a4paper,12pt]{article}
\usepackage{amsmath,amssymb,amsfonts,amsthm}
\usepackage{tikz}
\usepackage [utf8] {inputenc}
\usepackage [T2A] {fontenc} 
\usepackage[russian]{babel}
\usepackage{cmap}
\usepackage{listings}
\usepackage{color}

\definecolor{mygreen}{rgb}{0,0.6,0}
\definecolor{mygray}{rgb}{0.5,0.5,0.5}
\definecolor{mymauve}{rgb}{0.58,0,0.82} 

\lstdefinestyle{customc}{
	belowcaptionskip=1\baselineskip,
	breaklines=true,
	frame=L,
	xleftmargin=\parindent,
	language=Python,
	showstringspaces=false,
	basicstyle=\footnotesize\ttfamily,
	keywordstyle=\bfseries\color{green!40!black},
	commentstyle=\itshape\color{purple!40!black},
	identifierstyle=\color{blue},
	stringstyle=\color{orange},
}


\newtheorem{defenition}{Определение}
\newtheorem{theorem}{Теорема}
\newtheorem{problem}{Задача}

\lstset{escapechar=@,style=customc}

% Так ссылки в PDF будут активны
\usepackage[unicode]{hyperref}

% вы сможете вставлять картинки командой \includegraphics[width=0.7\textwidth]{ИМЯ ФАЙЛА}
% получается подключать, как минимум, файлы .pdf, .jpg, .png.
\usepackage{graphicx}
% Если вы хотите явно указать поля:
\usepackage[margin=1in]{geometry}
% Или если вы хотите задать поля менее явно (чем больше DIV, тем больше места под текст):
% \usepackage[DIV=10]{typearea}

\usepackage{fancyhdr}

\newcommand{\bbR}{\mathbb R}%теперь вместо длинной команды \mathbb R (множество вещественных чисел) можно писать короткую запись \bbR. Вместо \bbR вы можете вписать любую строчку букв, которая начинается с '\'.
\newcommand{\eps}{\varepsilon}
\newcommand{\bbN}{\mathbb N}
\newcommand{\dif}{\mathrm{d}}

\pagestyle{fancy}
\makeatletter % сделать "@" "буквой", а не "спецсимволом" - можно использовать "служебные" команды, содержащие @ в названии
\fancyhead[L]{\footnotesize Информатика\\ Внимание: конспект не проверялся преподавателями —- всегда используйте рекомендуемую литературу при подготовке к экзамену!}%Это будет написано вверху страницы слева
\fancyhead[R]{}
\fancyfoot[L]{}%имя автора будет написано внизу страницы слева
\fancyfoot[R]{\thepage}%номер страницы —- внизу справа
\fancyfoot[C]{ Замечания и предложения просьба писать на \href{mailto:pulsar@phystech.edu}{pulsar@phystech.edu}}%по центру внизу страницы пусто

\renewcommand{\maketitle}{%
	\noindent{\bfseries\scshape\large\@title\ \mdseries\upshape}\par
	\noindent {\large\itshape\@author}
	\vskip 2ex}
\makeatother
\def\dd#1#2{\frac{\partial#1}{\partial#2}}
\def\ddd#1#2#3{\frac{\partial^2#1}{\partial#2\partial#3}}

\title{Графы. III} 
\author{Тимофей Хирьянов} 
\date{27 февраля 2016 г.}

\begin{document}
	\maketitle
	\section{Задача Эйлера}
	\begin{problem}
		Задача о кёнигсбергских мостах (нем. Königsberger Brückenproblem) --- старинная математическая задача, в которой спрашивалось, как можно пройти по всем семи мостам Кёнигсберга, не проходя ни по одному из них дважды. 
	\end{problem}
		\begin{defenition}
			Эйлеров цикл — это цикл графа, проходящий через каждое ребро (дугу) графа ровно по одному разу.
		\end{defenition}
		\begin{defenition}
			Эйлеров граф — граф, содержащий эйлеров цикл.
		\end{defenition}
		\begin{defenition}
			Полуэйлеров граф — граф, содержащий эйлеров путь (цепь).
		\end{defenition}
	
	
	\begin{theorem}
		Граф является Эйлеровым $\Leftrightarrow$ кратность всех вершин четная.
	\end{theorem}
	\begin{theorem}
		Граф является полуэйлеровым $\Leftrightarrow$ кратность всех вершин четная, кроме двух.
	\end{theorem}
	
	Доказательство этих двух теорем опустим.
	
		
	\section{Задача Гамильтона}
	\begin{problem}
		Задача Гамильтона --- посетить каждую вершину ровно один раз.
	\end{problem}	
	\begin{defenition}
		Гамильтонов граф --- граф, содержащий гамильтонов цикл.
	\end{defenition}
	\begin{defenition}
		Гамильтонов цикл --- цикл, содержащий все вершины данного графа.
	\end{defenition}
	\begin{defenition}
		Полугамильтонов граф --- граф, содержащий гамильтонов путь.
	\end{defenition}
	
	\section{Задача о китайском почтальоне}
		\begin{problem}
			Задача о китайском почтальоне --- посетить все ребра не менее одного раза, при этом так, чтобы суммарная длина траектории путь была минимальна.
		\end{problem}
		Чем эта задача похожа на задачу Эйлера? Тем, что мы посещаем ребра, а не вершины, каждое ребро обязаны посетить хотя бы один раз. 
		Очевидно, что в случае если граф Эйлеров, то задача имеет однозначное решение: мы посещаем каждое ребро два раза. Но задача почтальона несколько шире --- мы можем посетить каждое ребро \emph{как минимум один раз}.
		То есть, по одному ребру можно проходить два раза. И вот тут возникает вопрос: по какому проходить два раза так, чтобы путь был \emph{наименьшим}. То есть наша задача --- сделать из неэйлерового графа хотя бы полуэйлеров.
		Эта задача нетривиальна.
		
	\section{Задача коммивояжера}
		\begin{problem}
			Найти оптимальный по длине гамильтонов цикл.
		\end{problem}
		
		Заметим, что в полносвязном графе задача Гамильтона не возникает, но имеет место задача коммивояжера.
		
		\subsection{Метод полного перебора}
			Можно перебрать все возможные пути, а потом из них выбрать наименьший.Этот алгоритм выполняется за $O(n!)$. 
			Приведем решение задачи коммивояжера:
			\lstinputlisting[language=Python]{voyajer.py}




\end{document}
