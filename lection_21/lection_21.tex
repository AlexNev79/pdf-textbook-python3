\documentclass[a4paper,12pt]{article}
\usepackage{amsmath,amssymb,amsfonts,amsthm}
\usepackage{tikz}
\usepackage [utf8] {inputenc}
\usepackage [T2A] {fontenc} 
\usepackage[russian]{babel}
\usepackage{cmap}
\usepackage{listings}
\usepackage{color}

\definecolor{mygreen}{rgb}{0,0.6,0}
\definecolor{mygray}{rgb}{0.5,0.5,0.5}
\definecolor{mymauve}{rgb}{0.58,0,0.82} 

\lstdefinestyle{customc}{
	belowcaptionskip=1\baselineskip,
	breaklines=true,
	frame=L,
	xleftmargin=\parindent,
	language=Python,
	showstringspaces=false,
	basicstyle=\footnotesize\ttfamily,
	keywordstyle=\bfseries\color{green!40!black},
	commentstyle=\itshape\color{purple!40!black},
	identifierstyle=\color{blue},
	stringstyle=\color{orange},
}


\newtheorem{defenition}{Определение}
\newtheorem{theorem}{Теорема}
\newtheorem{problem}{Задача}

\lstset{escapechar=@,style=customc}

% Так ссылки в PDF будут активны
\usepackage[unicode]{hyperref}

% вы сможете вставлять картинки командой \includegraphics[width=0.7\textwidth]{ИМЯ ФАЙЛА}
% получается подключать, как минимум, файлы .pdf, .jpg, .png.
\usepackage{graphicx}
% Если вы хотите явно указать поля:
\usepackage[margin=1in]{geometry}
% Или если вы хотите задать поля менее явно (чем больше DIV, тем больше места под текст):
% \usepackage[DIV=10]{typearea}

\usepackage{fancyhdr}

\newcommand{\bbR}{\mathbb R}%теперь вместо длинной команды \mathbb R (множество вещественных чисел) можно писать короткую запись \bbR. Вместо \bbR вы можете вписать любую строчку букв, которая начинается с '\'.
\newcommand{\eps}{\varepsilon}
\newcommand{\bbN}{\mathbb N}
\newcommand{\dif}{\mathrm{d}}

\pagestyle{fancy}
\makeatletter % сделать "@" "буквой", а не "спецсимволом" - можно использовать "служебные" команды, содержащие @ в названии
\fancyhead[L]{\footnotesize Информатика\\ Внимание: конспект не проверялся преподавателями —- всегда используйте рекомендуемую литературу при подготовке к экзамену!}%Это будет написано вверху страницы слева
\fancyhead[R]{}
\fancyfoot[L]{}%имя автора будет написано внизу страницы слева
\fancyfoot[R]{\thepage}%номер страницы —- внизу справа
\fancyfoot[C]{ Замечания и предложения просьба писать на \href{mailto:pulsar@phystech.edu}{pulsar@phystech.edu}}%по центру внизу страницы пусто

\renewcommand{\maketitle}{%
	\noindent{\bfseries\scshape\large\@title\ \mdseries\upshape}\par
	\noindent {\large\itshape\@author}
	\vskip 2ex}
\makeatother
\def\dd#1#2{\frac{\partial#1}{\partial#2}}
\def\ddd#1#2#3{\frac{\partial^2#1}{\partial#2\partial#3}}

\title{Поиск подстроки в строке} 
\author{Тимофей Хирьянов} 
\date{27 февраля 2016 г.}

\begin{document}
	\maketitle
	Рассмотрим наивный поиск и алгоритм Рабина-Карпа.
	В наивном поиске поиск осуществляется простым наложением паттерна на текст, последовательно, начиная с первого символа. 
	
	Алгоритм Рабина-Карпа пытается ускорить проверку эквивалентности образца с подстроками в тексте, используя хеш-функцию. Хеш-функция — это функция, преобразующая каждую строку в числовое значение, называемое хеш-значением (хеш); например, мы можем иметь хеш от строки «hello» равным 5. Алгоритм использует тот факт, что если две строки одинаковы, то и их хеш-значения также одинаковы. Таким образом, всё что нам нужно, это посчитать хеш-значение искомой подстроки и затем найти подстроку с таким же хеш-значением.
	
	Однако существуют две проблемы, связанные с этим. Первая состоит в том, что, так как существует очень много различных строк, между двумя различными строками может произойти коллизия — совпадение их хешей. В таких случаях необходимо посимвольно проверять совпадение самих подстрок, что занимает достаточно много времени, если данные подстроки имеют большую длину. При использовании достаточно хороших хеш-фунцкий коллизии случаются крайне редко, в результате среднее время поиска оказывается невелико.
	
	Еще одна проблема заключается в пересчете хеша. При навном пересчете затрачивается такое же время, как у более простых алгоритмов. Здесь же реализован кольцевой пересчет хеша, т.е. следующий символ добавляется в хеш, а последний убирается.
	\lstinputlisting[language=Python]{7-2.py}
	
	
	
	
\end{document}
