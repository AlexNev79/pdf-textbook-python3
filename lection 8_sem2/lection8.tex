\documentclass[a4paper,12pt]{article}
\usepackage{amsmath,amssymb,amsfonts,amsthm}
\usepackage{tikz}
\usepackage [utf8] {inputenc}
\usepackage [T2A] {fontenc} 
\usepackage[russian]{babel}
\usepackage{cmap}
\usepackage{listings}
\usepackage{color}

\definecolor{mygreen}{rgb}{0,0.6,0}
\definecolor{mygray}{rgb}{0.5,0.5,0.5}
\definecolor{mymauve}{rgb}{0.58,0,0.82} 

\lstdefinestyle{customc}{
	belowcaptionskip=1\baselineskip,
	breaklines=true,
	frame=L,
	xleftmargin=\parindent,
	language=Python,
	showstringspaces=false,
	basicstyle=\footnotesize\ttfamily,
	keywordstyle=\bfseries\color{green!40!black},
	commentstyle=\itshape\color{purple!40!black},
	identifierstyle=\color{blue},
	stringstyle=\color{orange},
}


\newtheorem{defenition}{Определение}
\newtheorem{theorem}{Теорема}
\newtheorem{problem}{Задача}

\lstset{escapechar=@,style=customc}

% Так ссылки в PDF будут активны
\usepackage[unicode]{hyperref}

% вы сможете вставлять картинки командой \includegraphics[width=0.7\textwidth]{ИМЯ ФАЙЛА}
% получается подключать, как минимум, файлы .pdf, .jpg, .png.
\usepackage{graphicx}
% Если вы хотите явно указать поля:
\usepackage[margin=1in]{geometry}
% Или если вы хотите задать поля менее явно (чем больше DIV, тем больше места под текст):
% \usepackage[DIV=10]{typearea}

\usepackage{fancyhdr}

\newcommand{\bbR}{\mathbb R}%теперь вместо длинной команды \mathbb R (множество вещественных чисел) можно писать короткую запись \bbR. Вместо \bbR вы можете вписать любую строчку букв, которая начинается с '\'.
\newcommand{\eps}{\varepsilon}
\newcommand{\bbN}{\mathbb N}
\newcommand{\dif}{\mathrm{d}}

\pagestyle{fancy}
\makeatletter % сделать "@" "буквой", а не "спецсимволом" - можно использовать "служебные" команды, содержащие @ в названии
\fancyhead[L]{\footnotesize Информатика}%Это будет написано вверху страницы слева
\fancyhead[R]{\footnotesize МФТИ}
\fancyfoot[L]{\footnotesize \@author}%имя автора будет написано внизу страницы слева
\fancyfoot[R]{\thepage}%номер страницы —- внизу справа
\fancyfoot[C]{}%по центру внизу страницы пусто

\renewcommand{\maketitle}{%
	\noindent{\bfseries\scshape\large\@title\ \mdseries\upshape}\par
	\noindent {\large\itshape\@author}
	\vskip 2ex}
\makeatother
\def\dd#1#2{\frac{\partial#1}{\partial#2}}
\def\ddd#1#2#3{\frac{\partial^2#1}{\partial#2\partial#3}}

\title{Автоматы. Машина Тьюринга. Вычислимость.} 
\author{Тимофей Хирьянов} 
\date{12 апреля 2016 г.}

\begin{document}
	\maketitle
	\section{Машина Тьюринга}
	\begin{defenition}
	Машина Тьюринга (МТ) — математическая абстракция, представляющая вычислительную машину общего вида. 
	\end{defenition}
	Была предложена Аланом Тьюрингом в 1936 году для формализации понятия алгоритма.

	Машина Тьюринга является расширением модели конечного автомата и, согласно тезису Чёрча — Тьюринга, способна имитировать (при наличии соответствующей программы) любую машину, действие которой заключается в переходе от одного дискретного состояния к другому.

	В состав Машины Тьюринга входит бесконечная в обе стороны лента, разделённая на ячейки, и управляющее устройство с конечным числом состояний.

	Управляющее устройство может перемещаться влево и вправо по ленте, читать и записывать в ячейки символы некоторого конечного алфавита. Выделяется особый пустой символ, заполняющий все клетки ленты, кроме тех из них (конечного числа), на которых записаны входные данные.

	В управляющем устройстве содержится таблица переходов, которая представляет алгоритм, реализуемый данной Машиной Тьюринга. Каждое правило из таблицы предписывает машине, в зависимости от текущего состояния и наблюдаемого в текущей клетке символа, записать в эту клетку новый символ, перейти в новое состояние и переместиться на одну клетку влево или вправо. Некоторые состояния Машины Тьюринга могут быть помечены как терминальные, и переход в любое из них означает конец работы, остановку алгоритма.
	
	\section{Конечный автомат}
	\begin{defenition}
		Конечный автомат — абстрактный автомат, число возможных внутренних состояний которого конечно.
	\end{defenition}
	Реализуем алгоритм поиска подстроки <<abcd>> в строке.
	\lstinputlisting[language=Python]{au.py}
	
	
	
\end{document}
