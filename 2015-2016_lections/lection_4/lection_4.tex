% !TeX spellcheck = russian_english
% !TeX encoding = UTF-8
\documentclass[a4paper, fleqn]{article}

\usepackage{indentfirst} % Красная строка
\usepackage[T2A]{fontenc} % Поддержка русских букв
\usepackage[utf8]{inputenc} % Кодировка utf8
\usepackage[russian]{babel} % руссификация
\usepackage{amssymb} % дополнительные символы
\usepackage{textcomp} % дополнительные текстовые символы
\usepackage{amsmath} % матрицы
\usepackage{listings}

% \input{listings-python.prf}
%\ifx\pdfoutput\undefined
%\usepackage{graphicx}
%\else
%\usepackage[pdftex]{graphicx}
%\fi

\textwidth = 16 cm
\oddsidemargin= 0 cm
\evensidemargin= 1 cm


% \abovedisplayskip = 0 pt %.2\abovedisplayskip
% \belowdisplayskip = .2\belowdisplayskip
% \abovedisplayshortskip=.2\abovedisplayshortskip
% \belowdisplayshortskip=.2\belowdisplayshortskip
% \topsep= 0 pt

\setlength{\mathindent}{1.2 cm}
% \setlength{\topsep}{0 pt}
% \setlength{\abovedisplayskip}{0 pt}

% \clubpenalty = 5000 % запрет висячих строк
% \widowpenalty = 5000
\binoppenalty=10000
\relpenalty=10000

% собственные команды и окружения

% \newcommand*{\spt}[2][ени]{простран\-ств#2-врем#1}
% \newcommand*{\ks}[1]{$K$-систем#1} % учитывать пробел!
% \newcommand*{\kss}[1]{$K'$-систем#1} % учитывать пробел!
\newenvironment{example}[1][]{\medskip \noindent \textbf{Пример. #1}\par \nopagebreak}{\medskip \par} % окружение-"пример"
\newcommand*{\py}{Python}

\lstset{language=Python, tabsize=4, language=Python}


\title{Лекция \textnumero\,4}
	% {\huge \vspace{3 cm}}}

\author{Т.\,Ф. Хирьянов}

\date{}

\begin{document}
	\section*{Алгебра логики}
	\subsection*{Составные высказывания}
	
	В логике Аристотеля существуют два состояния: \emph{истина} и \emph{ложь}, которые удобно обозначать соответственно как 1 и 0. 
	Изначально рассматривалось содержание высказываний. Однако впоследствии утвердился подход, в котором существуют атомарные (простые) высказывания, не разбивающиеся на отдельные. Их истинность зависит от сторонних знаний, и в конкретном рассмотрении атомарное выражение принимается истинным либо ложным аксиоматически. Так например, выражение 
	$1+1=1$
	можно считать ложным с точки зрения привычной арифметики либо истинным, если рассматривать алгебру, где это возможно. 
	
	Предметом же рассмотрения алгебры логики являются \emph{составные высказывания}. Для их образования из простых высказываний используются \emph{логические операции}. Они выполняются в определенном порядке, соблюдая приоритетность, которая убывает следующим образом.
	\begin{enumerate}
		\item \emph{не А} (логическое отрицание). \\
		Эта операция имеет самый большой приоритет и применяется к одному высказыванию. На письме отрицание А (не А) записывается одним из двух равноправных способов:
		\[\neg A \qquad \qquad \bar{A}\]
		\item \emph{A или B} (логическое умножение)\\
		Такая операция применяется к двум выражениям. Ее запись также имеет альтернативу.
		\[A \wedge B \qquad A \cdot B\]
		\item \emph{A или B} (логическое сложение)\\
		Записывается аналогично.
		\[A \vee B \qquad A + B\]
		\item \emph{импликация} \\
		Эта операция может быть выражена с помощью предшествующих трех, однако из-за частой встречаемости обозначается специальным образом.
		\[A \rightarrow B \qquad A \Rightarrow B\]
		\item \emph{эквиваленция}\\
		Является прямой и обратной импликацией. 
		\[A \leftrightarrow B \qquad A \Leftrightarrow B\]
		\item \emph{либо} (исключающее или)\\
		Обозначается так.
		\[A \oplus B\]
		
	\end{enumerate}
	
	%10:40
	По своей сути эти операторы являются логическими функциями и, как любые функции, отображают множество, называемое областью определения,  на множество, являющееся областью значений. Так как область определения в данном случае состоит из двух элементов -- правды~(1) и лжи~(0) -- то можно сопоставить каждому элементу соответствующее значение функции (тоже 1 или~0). Получается что для одного переменного существует $2*2=4$ функций, а для двух аргументов -- $(2*2)*2=8$ функций. Удобно записать это в виде таблицы. Для введенных ранее операций они будут выглядеть следующим образом.
	
	\begin{tabular}{c|c}
		A&  not A\\ 
		\hline  
		0&  1\\ 
		1&  0\\  
	\end{tabular} \qquad \qquad
	%
	\begin{tabular}{cc|c|c|c|c|c}
		A&  B&  $A \cdot B$&  $A+B$&  $A \oplus B$& $A\Rightarrow B$&  $A \Leftrightarrow B$\\ 
		\hline
		0&  0&  0&  0&  0&	1&  1\\ 
		0&  1&  0&  1&  1&	1&  0\\ 
		1&  0&  0&  1&  1&	0&  0\\ 
		1&  1&  1&  1&  0&	1&  1\\ 
	\end{tabular} 
	
	Так $A \cdot B$ возвращает истину только когда оба аргумента истинны, а $A+B$ наоборот возвращает ложь только когда оба аргумента ложны. Исключающее или $A \oplus B$ похоже на обычное или, однако в отличие от него возвращает ложь в случае истинности обоих аргументов; значение данной операции равно правде только в том случае, если один из аргументов является правдой, а другой обязательно ложью. 
	Эквиваленция верна, когда оба аргумента равны друг другу.
	
	Особого пояснения заслуживает импликация (следствие). Очевидно, что если из истинного утверждения следует истинна, то рассуждения верны, т.е.~истинны ($1\Rightarrow 1 = 1$). Также истинно рассуждение, если из ложного утверждения получается ложное. Например, $2+2 = 5$ ложно, откуда следует, что и $4+4=10$ тоже ложно. Может показаться противоречивым, но из ложного суждения может следовать истинное. Так, $2+2 = 5$ ложно. Утверждение $5=2+2$ также ложно и является справедливым следствием первого утверждения (так как$0 \Rightarrow 0 = 1$). Так как эти утверждения ложны одновременно, то их логическое произведение возвращает ложь($0\cdot 0 = 0$). Однако из этих двух утверждений вместе следует, что $2+2 = 2+2$, а это истинное выражение, т.е. $0 \Rightarrow 1 = 1$. Импликация возвращает ложь только в случае, когда из верного утверждения следует ложное. Действительно, если из утверждения $2+2=4$ получается, что, например, $3+3=5$, то рассуждения, очевидно, не верны ($1 \Rightarrow 0 = 0$). Анализируя полученные значения для импликации, можно заметить, что из ложного утверждения может следовать как правда, так и ложь. Именно поэтому в системе аксиом не должно быть противоречивых утверждений, так как тогда следствия из них не всегда будут верны. 
	
	% 26:20 логическое "не равно" эквивалентно исключающему или.
	
	\subsection*{Нормальная дизъюнктивная форма}
	
	Из указанных функций некоторые выражаются через другие. Из-за этого возникает вопрос, какого количества из этих операций будет достаточно для описания произвольной логической функции любого количества аргументов. Теория показывает, что трех из них (не, и, или) достаточно для этого, нужно только привести функцию к \emph{нормальной дизъюнктивной форме}, т.е. к логической сумме специально сконструированных произведений, равных истине только в одном из случаев, когда функция также истинна. Значит, таких произведений будет столько, сколько раз функция принимает значение 1 в ее таблице истинности.
	
	\begin{tabular}{ccc|c|c|c|c}
		A&  B&  C&  f(A,B,C)&  $\bar{A}\cdot \bar{B}\cdot C$&  $\bar{A}\cdot B \cdot C$&  $A\cdot B \cdot \bar{C}$\\
		\hline
		0&  0&  0&  0&  0&  0&  0\\ 
		0&  0&  1&  1&  1&  0&  0\\ 
		0&  1&  0&  0&  0&  0&  0\\ 
		0&  1&  1&  1&  0&  1&  0\\ 
		1&  0&  0&  0&  0&  0&  0\\ 
		1&  0&  1&  0&  0&  0&  0\\ 
		1&  1&  0&  1&  0&  0&  1\\ 
		1&  1&  1&  0&  0&  0&  0\\ 
	\end{tabular} 
	
	Тогда нормальная дизъюнктивная форма для приведенной функции будет равна следующей величине.
	\[f(A, B, C) = \bar{A}\cdot \bar{B}\cdot C + \bar{A}\cdot B \cdot C + A\cdot B \cdot \bar{C}\]
	Из нее видно, что логического отрицания, умножения и сложения достаточно, чтобы представить логическую функцию в виде такой формы.
	
	\subsection*{Законы алгебры логики}
	%39:45
	Очевидные соотношения:
	\begin{itemize}
		\item $A+1 = 1$
		\item $A+0=A$
		\item $A\cdot1=A$
		\item $A\cdot0=0$
	\end{itemize}
	Простые законы:
	\begin{itemize}
		\item $A\cdot A = A$ \qquad $ A + A = A$(законы повторения)
		\item $A\cdot\bar{A}=0$ \qquad $A + \bar{A} = 1$ (законы <<третьего лишнего>>)
		\item $A\cdot(A+B)=A$ $A+A\cdot B=A$ (закон поглощения)\\
		Действительно, подставив вместо А 1, а затем 0, получаем с помощью очевидных соотношений верные утверждения.
	\end{itemize}
	Свойства математических операций.
		\begin{itemize}
			\item \begin{tabbing}
				\hspace{6cm}\=\kill
				$A\cdot B=B\cdot A$ \> $A+B=B+A$ (коммутативность)
			\end{tabbing}
			\item \begin{tabbing}
				\hspace{6cm}\=\kill
				$A\cdot(B\cdot C)= (A\cdot B)\cdot C = A\cdot B\cdot C$ \> $A+(B+C)=(A+B)+C=A+B+C$ (ассоциативность)
			\end{tabbing}
			\item \begin{tabbing}
				\hspace{6cm}\=\kill
				$A\cdot(B+C)=A\cdot B+A\cdot C$ \> $A+B\cdot C = (A+B)\cdot(A+C)$  (дистрибутивность)
			\end{tabbing}	
		\end{itemize}
	Законы отрицания.
	\begin{itemize}
		\item $\bar{\bar{A}}=A$
		\item $\neg(A+B)=\bar{A}\cdot\bar{B}$
		\item $\neg(A\cdot B)=\bar{A}+\bar{B}$
	\end{itemize}
	Представления операторов.
	\begin{itemize}
		\item $A \Rightarrow B = \bar{A} + B$
		\item $A \Leftrightarrow B = A\cdot B + \bar{A}\cdot \bar{B}$
		\item $A B = \bar{A}\cdot B + A\cdot B$
	\end{itemize}
	%51:42	
	
	\subsection*{Определение четверти координатной плоскости}
	Рассмотрим часто встречающуюся логическую задачу -- определение множества, которому принадлежит элемент. В качестве множеств будут выступать координатные четверти плоскости, а элементом будет точка с координатами x~и~y. Заметим, что если ордината точки положительна, то точка может принадлежать первой или второй четверти. Чтобы выбрать между ними, нужно проверить знак абсциссы. Если она положительна, то точка из первой четверти, если отрицательна, то из второй. 
	\texttt{
		\begin{tabbing}
			\hspace{8mm}\=\hspace{8mm}\=\hspace{2cm}\=\kill
			x = int(input())\\
			y = int(input())\\
			if y > 0:\\
			\> if x > 0:\\
			\> \> print('1')\\
			\> else:\\
			\> \> print('2')\\
			else:\\
			\> if x > 0:\\
			\> \> print('4')\\
			\> else:\\
			\> \> print('3')
		\end{tabbing}
	}
	Заметим, что приведенное выше решение неверно в случае принадлежности точки осям координат. Это пример того, что перед написанием кода важно предварительно определить какими могут быть входные данные, будет ли задача корректной для них, какие случаи необходимо рассмотреть, чтобы решение было исчерпывающим.  
	\subsection*{Каскадная условная конструкция}
	При решении многих задач возникает типичная ситуация: входное данное в большинстве случаев (например, когда оно отрицательно) правильно обрабатывается некоторым алгоритмом.
	
	Если множества, которым может принадлежать переменная, являются вложенными и их можно однозначно разделить по значению некоторых параметров, то удобно использовать каскадную условную конструкцию else ... if, которая ввиду частого употребления получила в Python укороченный синтаксис: можно просто записать elif. В приведенном ниже примере в качестве вложенных множеств выступают ($ x < 0 $), ($ x \leqslant  0 $), ($ x < 1 $), ($ x \leqslant 1 $). 
	\texttt{
		\begin{tabbing}
			\hspace{8mm}\=\hspace{8mm}\=\hspace{2cm}\=\kill
			x = int(input())\\
			if x < 0:\\
			\> print('negative')\\
			elif x == 0:\\
			\> print('null')\\
			elif x < 1:\\
			\> print('0 < x < 1')\\
			elif x == 1:\\
			\> print('1')\\
			else: \> \> \# x > 1\\
			\> print('x > 1')
		\end{tabbing}
	}
	\subsection*{Поиск максимального значения}
	Для поиска наибольшего элемента в некотором наборе нужно задать начальное значение временного максимума, а затем сравнивать с ним каждый элемент. При нахождении элемента большего временного максимума, последний обновляется. В качестве начального значения лучше выбрать значение одного из элементов.
	\texttt{
		\begin{tabbing}
			\hspace{8mm}\=\hspace{8mm}\=\hspace{8mm}\=\hspace{2cm}\=\kill
			x = int(input())\\
			tmp\_max = x\\
			while x!=0:\\
			\> if x > tmp\_max: \> \> \> \# tmp\_max = max(x, tmp\_max)\\
			\> \> tmp\_max = x\\
			\> x = int(input())
		\end{tabbing}
	}
	Допустим, что требуется найти три наибольших числа. Тогда использовать нужно уже три временных максимума, а сравнение значений элементов с ними теперь происходит внутри каскадной конструкции. С помощью кортежей обновление максимумов можно сделать очень просто. 
	\texttt{
		\begin{tabbing}
			\hspace{8mm}\=\hspace{8mm}\=\hspace{2cm}\=\kill
			x = int(input())\\
			\# задание начальных значений m1 > m2 > m3\\
			while x!=0:\\
			\> if x > m1:\\
			\> \> m1, m2, m3 = x, m1, m2\\
			\> elif x > m2:\\
			\> \> m2, m3 = x, m2\\
			\> elif x > m3:\\
			\> \> m3 = x\\
			\> x int(input())
		\end{tabbing}
	}
	
	\subsection*{Тест простоты числа}
	В приведенной ниже реализации используется тот факт, что делитель числа не может превосходить квадратный корень последнего. 
	\texttt{
		\begin{tabbing}
			\hspace{8mm}\=\hspace{8mm}\=\hspace{8mm}\=\hspace{2cm}\=\kill
			def isPrime(x):\\
			\> "x is a prime"\\
			\> delitel = 2\\
			\> while delitel**2 < x:\\
			\> \> if x \% delitel == 0:\\
			\> \> \> return False\\
			\> \> delitel += 1\\
			\> return True
		\end{tabbing}
	}
	\subsection*{Разложение числа на множители}
	Для решения данной задачи необходимо проверить равен ли остаток от деления выбранного числа на некоторый делитель. Если это так, то дальше на делимость можно исследовать частное, иначе нужно переходить на следующий возможный делитель.
	\texttt{
		\begin{tabbing}
			\hspace{8mm}\=\hspace{8mm}\=\hspace{8mm}\=\hspace{2cm}\=\kill
			x = int(input())\\
			delitel = 2\\
			while x > 1:\\
			\> if x \% delitel == 0:\\
			\> \> print (delitel)\\
			\> \> x //= delitel\\
			\> else:\\
			\> \> delitel += 1
		\end{tabbing}
	}
	
\end{document}
