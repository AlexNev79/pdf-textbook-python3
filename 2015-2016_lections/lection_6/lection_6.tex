% !TeX spellcheck = russian_english
% !TeX encoding = UTF-8
\documentclass[a4paper, fleqn]{article}

\usepackage{indentfirst} % Красная строка
\usepackage[T2A]{fontenc} % Поддержка русских букв
\usepackage[utf8]{inputenc} % Кодировка utf8
\usepackage[russian]{babel} % руссификация
\usepackage{amssymb} % дополнительные символы
\usepackage{textcomp} % дополнительные текстовые символы
\usepackage{amsmath} % матрицы
\usepackage{listings}
% листинги
\lstset{language=Python, tabsize=4, language=Python}


\textwidth = 16 cm
\oddsidemargin= 0 cm
\evensidemargin= 1 cm


% \abovedisplayskip = 0 pt %.2\abovedisplayskip
% \belowdisplayskip = .2\belowdisplayskip
% \abovedisplayshortskip=.2\abovedisplayshortskip
% \belowdisplayshortskip=.2\belowdisplayshortskip
% \topsep= 0 pt

% \setlength{\mathindent}{1.2 cm}
% \setlength{\topsep}{0 pt}
% \setlength{\abovedisplayskip}{0 pt}

% \clubpenalty = 5000 % запрет висячих строк
% \widowpenalty = 5000
\binoppenalty=10000
\relpenalty=10000

% собственные команды и окружения

\newenvironment{example}[1][]{\medskip \noindent \textbf{Пример. #1}\par \nopagebreak}{\medskip \par} % окружение-"пример"


\title{Лекция \textnumero\,6}
% {\huge \vspace{3 cm}}}

\author{Т.\,Ф. Хирьянов}

\date{}

\begin{document}
	\maketitle
	
\subsection*{Структурное программирование}

Небольшие программы, имеющие порядка ста строк кода, обычно просты в понимании, однако чем больше она становится, тем сложнее уловить ее принцип работы, понять, что обозначает та или иная переменная, найти и исправить ошибку. Именно поэтому код программы должен быть структурирован.
% 5:55

\subsection*{Базовые принципы структурного программирования}

% 09:25
\begin{itemize}
	\item Программа состоит из 
	\begin{enumerate}
		\item \emph{последовательного исполнения}
		\item \emph{ветвлений}
		\item \emph{циклов}
	\end{enumerate}
	\item Повторяющиеся участки кода оформляют в виде \emph{функций}
	\item Разработка программы осуществляется пошагово \emph{сверху-вниз}
\end{itemize}

В качестве иллюстрации можно рассмотреть следующую задачу. На вход поступают строки с автомобильным номерам и величиной скорости, с которой транспортное средство проезжает мимо поста ГИБДД. 

Патрульный останавливает машины, превысившие скорость 60 км/ч, и озвучивает размер взятки, которая зависит от <<привилегированности>> номера: если в номере все три цифры разные, то 100 рублей, если есть две одинаковые, то 200, если три, то 1000. При этом, когда в конце рабочего дня приезжает начальник, его не уличают в превышении скорости, даже если это случилось. Программа должна вывести <<зарплату>> постового за день.
%
\texttt{
	\begin{tabbing}
		\hspace{8mm}\=\hspace{8mm}\=\hspace{4cm}\=\kill
		A238BE	73	$\longrightarrow$ 100\\
		B202CC	84	$\longrightarrow$	200\\
		B555PH	71	$\longrightarrow$	1000\\
		...\\
		A999AA	100
	\end{tabbing} 
}

Приведенный ниже код данной программы написан с соблюдением положений структурного программирования. 
%
\texttt{
	\begin{tabbing}
		\hspace{8mm}\=\hspace{8mm}\=\hspace{8mm}\=\hspace{4cm}\=\kill
		def solve\_task():\\
		\> print(count\_salary())\\
		\\
		def count\_salary():\\
		\> salary = 0\\
		\> licence\_num, speed = input().split()\\
		\> while not chief(licence\_num):\\
		\> \> if float(speed) > 60:\\
		\> \> \> salary += count\_tax(licence\_num)\\
		\> \> licence\_num, speed = input().split()\\
		\> return salary\\
		\\
		def chief(licence\_num):\\
		\> return licence\_num == "A999AA"\\
		\\
		def count\_tax(licence\_num):\\
		\> return 0		\# FIXME
	\end{tabbing} 
}
Действительно, в нем код оформлен в функции, имеющие интуитивно понятные названия. Функция \texttt{solve\_task} выводит на экран зарплату. При этом она не подсчитывает ее - это выполняет функция \texttt{count\_salary()}. В последней используется переменная-счетчик \texttt{salary}, значение которой отражает текущий доход. В \texttt{licence\_num} и \texttt{speed} записываются соответственно номер и скорость автомобиля. Пока номера считываются в цикле, происходит анализ того, не принадлежит ли текущий номер начальнику и какой размер взятки запросить. Это осуществляется посредством вызова функций \texttt{chief} и \texttt{count\_tax}. 

Важно отметить, что вместо вызова \texttt{chief} можно было явно прописать в условии \texttt{licence\_num == "A999AA"}, однако от этого бы пострадала ясность и понятность кода.

На этапе разработки можно лишь объявить некоторые функции, запрограммировав их так, что бы они возвращали корректное, хотя и неверное, значение (как в случае с \texttt{count\_tax}).
% лучше сноской
Для того чтобы впоследствии вернуться к доработке программы удобно пометить необходимые места кода комментарием \texttt{FIXME}. 

% 34:30
Прежде чем начинать программировать, нужно спроектировать программу: определить, как будет осуществляться взаимодействие между функциями(какие данные будут подаваться на вход и что будет возвращаться). Все это лучше всего прописать в текстовой документации к программе, однако предварительно можно просто указать в многострочном комментарии в начале тела функции. Например, таким образом.
% 41:00
%
\texttt{
	\begin{tabbing}
		\hspace{8mm}\=\hspace{8mm}\=\hspace{8mm}\=\hspace{4cm}\=\kill
		def count\_tax(licence\_num):	\\
		\> """ 2 numbers - 200\\
		\> \> 3 numbers - 1000\\
		\> \> else - 100 """\\	
		\> pass
	\end{tabbing} 
}
% 50:40
% 
\subsection*{Стек вызовов}

Допустим, что существует программа А, в процессе выполнения которой вызывается функция B. В таком случае работа программы А должна быть приостановлена и начато выполнение функции В. При этом по окончании работы последней программа А должна возобновить свою работу с того места, где она была прервана. Для этого необходимо сохранить в стеке \emph{адрес возврата}. 
Если теперь уже в процессе выполнения функции В была вызвана функция С, то выполняются аналогичные действия. При этом адрес возврата к В кладется <<сверху>> предыдущего. 

После того, как выполнение функции С подойдет к концу, команда \emph{return} обеспечит возврат к необходимому месту и удалит соответствующий адрес из стека. При этом переменные, которые завела функция С для своей работы являются локальными и по окончании ее работы также удаляются - это важно иметь в виду при проектировании программы.

% 56:00
\subsection*{Именованные параметры}

Если при вызове функции один или несколько из передаваемых параметров  в большинстве случаев имеют определенные значения, то удобно указать их явно в заголовке функции. Тогда они станут значениями по умолчанию и их можно будет не прописывать, при вызове функции. В приведенной ниже функции, меняющей пробел на другой разделитель \texttt{sep}, указано значение по умолчанию  последнего (\texttt{sep='.'}).
%
%
\texttt{
	\begin{tabbing}
		\hspace{8mm}\=\hspace{8mm}\=\hspace{8mm}\=\hspace{4cm}\=\kill
		def my\_print(s, sep='.'):\\
		\> res = ''\\
		\> for symbol in s:\\
		\> \> res += symbol + sep\\
		\> print(res)\\
		\\
		my\_print('Hello')
	\end{tabbing} 
}	
Также именованные параметры при вызове можно менять местами. Однако для этого нужно указывать их явно.
%
\texttt{
	\begin{tabbing}
		\hspace{8mm}\=\hspace{8mm}\=\hspace{8mm}\=\hspace{4cm}\=\kill
		def f(x, y):\\
		\> return x/y\\
		\\
		f(1, 2)	\> \> \> \# f(x=1, y=2)\\
		f(y=1, x=2)
	\end{tabbing} 
}	
В Python существует много \emph{итерируемых объектов}, по элементам которых можно пробегаться в цикле. Например, в приведенной ниже функции \texttt{x} пробегает все значения от нулевого до последнего элемента \texttt{A}, где \texttt{A} может являться как списком, так и строкой.
%
\texttt{
	\begin{tabbing}
		\hspace{8mm}\=\hspace{8mm}\=\hspace{8mm}\=\hspace{4cm}\=\kill
		m = 0\\
		for x in A:\\
		\> if x > m:\\
		\> \> m = x
	\end{tabbing} 
}
В данном примере слово <<Hello>> будет выведено на экран побуквенно.
%
\texttt{
	\begin{tabbing}
		\hspace{8mm}\=\hspace{8mm}\=\hspace{8mm}\=\hspace{4cm}\=\kill
		for symbol in 'Hello':\\
		\> print(symbol)
	\end{tabbing} 
}
% 1:02:00
\subsection*{Генерация списков}

В Python есть очень удобная возможность создавать списки с помощью обхода элементов итерируемого объекта. Для этого нужно в квадратных скобках указать следующее.\\
\texttt{new\_A = [f(x) for x in A]}\\
где \texttt{f(x)}  -- значение, сопоставляемое каждому \texttt{x} из итерируемого объекта \texttt{A}.

В приведенном ниже примере список В будет содержать квадраты элементов списка \texttt{A}. Более того можно задать определенное условие, при котором элементы попадут в новый список, как при генерации \texttt{C} (попадут только четные).
%
\texttt{
	\begin{tabbing}
		\hspace{8mm}\=\hspace{8mm}\=\hspace{8mm}\=\hspace{4cm}\=\kill
		A = [int(x) for x in input().split()]\\
		B = [x**2 for x in A]\\
		C = [x for x in A if x \% 2 == 0]
	\end{tabbing} 
}	
В Python можно создавать и двумерные (и более) списки. В приведенном ниже примере в списке А окажутся два элемента -- списки, содержащие по три элемента. Доступ  к элементу и его изменение осуществляется с помощью индексов.
%
\texttt{
	\begin{tabbing}
		\hspace{8mm}\=\hspace{8mm}\=\hspace{8mm}\=\hspace{4cm}\=\kill
		A = []\\
		A.append([1, 2, 3])\\
		A.append([4, 5, 6])\\
		\# A[1][2] == 6\\
		A[1][2] == 7
	\end{tabbing} 
}
Для списков существует операция повторения (*) с синтаксисом \texttt{x = [a]*N} (здесь \texttt{x} будет списком, в котором число а повторено \texttt{N} раз). При этом каждый элемент А будет ссылаться на объект <<а>>. Если же создать список списков так, как показано в примере, то при изменении элемента одного из внутренних списков, соответствующая величина а в других также поменяется.
%
\texttt{
	\begin{tabbing}
		\hspace{8mm}\=\hspace{8mm}\=\hspace{8mm}\=\hspace{4cm}\=\kill
		N = int(input())\\
		M = int(input())\\
		A = [0]*N\\
		B = [[0]*N]*M\\
	\end{tabbing} 
}	
Чтобы избежать данной проблемы удобно использовать генератор вложенных списков, в котором в качестве значения элемента генерируемого списка выступает другой список тоже генерируемый. Например, таким образом можно создать таблицу умножения.
%
\texttt{
	\begin{tabbing}
		\hspace{8mm}\=\hspace{8mm}\=\hspace{8mm}\=\hspace{4cm}\=\kill
		A = [[0]*N for i in range(M)]\\
		A = [[i*j for i in range(10)]\\
		\> for j in range(10)]
	\end{tabbing} 
}
%
\texttt{
	\begin{tabbing}
		\hspace{8mm}\=\hspace{8mm}\=\hspace{8mm}\=\hspace{4cm}\=\kill
		A = [[0]*N for i in range(M)]\\
		A = [[4*j + i for i in range(4)] for j in range(3)]
	\end{tabbing} 
}
Или такого рода таблицу.\\


\begin{tabular}{|c|c|c|c|}
	\hline \rule[-2ex]{0pt}{5.5ex}  0&  1&  2&  3\\ 
	\hline \rule[-2ex]{0pt}{5.5ex}  4&  5&  6&  7\\ 
	\hline \rule[-2ex]{0pt}{5.5ex}  8&  9&  10&  11\\ 
	\hline 
\end{tabular}
\\

Полиморфизм в Python заключается в том, что типы параметров могут быть различны. Главное, чтобы с данными типами корректно работали команды тела функции.
%
\texttt{
	\begin{tabbing}
		\hspace{8mm}\=\hspace{8mm}\=\hspace{8mm}\=\hspace{4cm}\=\kill
		def plus(a, b):\\
		return a + b\\
		\\
		plus (1, 2)\\
		plus (1.5, 7.5)\\
		plus ('ab', 'c')
	\end{tabbing} 
}
% 1:20:00
%
По этой же самой причине список можно использовать вместо логического выражения: если список пуст - это будут интерпретироваться как ложь, иначе -- правда. 
\texttt{
	\begin{tabbing}
		\hspace{8mm}\=\hspace{8mm}\=\hspace{8mm}\=\hspace{4cm}\=\kill
		A = []\\
		x = input()\\
		while x != '0':\\
		\> A.append(x)\\
		\> x = input()\\
		while A:\\
		\> print(A.pop())\\
	\end{tabbing} 
}

\end{document}
