% !TeX spellcheck = russian_english
% !TeX encoding = UTF-8
\documentclass[a4paper, fleqn]{article}

\usepackage{indentfirst} % Красная строка
\usepackage[T2A]{fontenc} % Поддержка русских букв
\usepackage[utf8]{inputenc} % Кодировка utf8
\usepackage[russian]{babel} % руссификация
\usepackage{amssymb} % дополнительные символы
\usepackage{textcomp} % дополнительные текстовые символы
\usepackage{amsmath} % матрицы
%\usepackage{pdfpages}
%\usepackage{pgfplots}
%\usepackage{pgfplotstable}
\usepackage{listings}
% листинги
\lstset{language=Python, tabsize=4, language=Python}


\textwidth = 16 cm
\oddsidemargin= 0 cm
\evensidemargin= 1 cm


% \abovedisplayskip = 0 pt %.2\abovedisplayskip
% \belowdisplayskip = .2\belowdisplayskip
% \abovedisplayshortskip=.2\abovedisplayshortskip
% \belowdisplayshortskip=.2\belowdisplayshortskip
% \topsep= 0 pt

% \setlength{\mathindent}{1.2 cm}
% \setlength{\topsep}{0 pt}
% \setlength{\abovedisplayskip}{0 pt}

% \clubpenalty = 5000 % запрет висячих строк
% \widowpenalty = 5000
\binoppenalty=10000
\relpenalty=10000

% собственные команды и окружения

\newenvironment{example}[1][]{\medskip \noindent \textbf{Пример. #1}\par \nopagebreak}{\medskip \par} % окружение-"пример"


\title{Лекция \textnumero\,5}
% {\huge \vspace{3 cm}}}

\author{Т.\,Ф. Хирьянов}

\date{}

\begin{document}
	\maketitle
	
	\section*{Работа со списками в Python 3}
	
	\subsection*{Кодировки символов}
	
	Американская стандартная система кодирования ASCII позволяла сопоставить распространенным печатным и непечатным символам соответствующие числовые коды. Изначально она включала в себя латинский алфавит. С помощью изменения старшего бита на 1 имелась возможность кодировать национальные символы. Из-за этого возникло очень много  различных национальных кодировок. Более того, такой подход не позволял писать международные письма, использующие одновременно несколько кодировок.
	 
	% 03:12
	Это привело к созданию международного стандарта  Unicode, который предлагал несколько кодировок, среди которых UTF-16, UTF-32. Первая из них позволяла кодировать $65532$ символа, а вторая приблизительно $4\cdot10^9$, причем в них на каждый символ отводилось одинаковое количество памяти. Из-за довольно большого размера таких равномерных кодировок была предложена новая UTF-8, которая являлась неравномерной, т.е. в ней разные символы имеют разную длину. Также это позволяет добавлять все новые и новые символы. Язык Python  ориентирован на стандарт Unicode, поэтому в строках можно использовать любые символы, если знать их код.
	
	\subsection*{Тип str}
	
	% 05:53
	Для разных языков программирования существуют разные подходы к реализации символьного типа. Например, в Pascal это тип \emph{char}, который  является символьным. В языке программирования Си существует тип с таким же названием, который является числовым.  В Python же не существует символьного типа, вместо этого есть тип <<строка>> (\emph{str}). Чтобы обратиться к  конкретному символу в строке необходимо указать его номер в качестве индекса. 
	\texttt{
		\begin{tabbing}
			\hspace{8mm}\=\hspace{8mm}\=\hspace{2cm}\=\kill
			s = 'Hello'\\
			s[0] == 'H'\\
			type(s[0]) == class<str>
		\end{tabbing}
	}
	Также символ можно понимать как срез строки длиной в один символ. При этом необходимо учитывать, что, например, в Pascal строки изменяемые. В Python это не так -- для того чтобы изменить строку, необходимо породить новый объект и затем сделать привязку к старому идентификатору. 
	
	\subsection*{Функции строк}
	
	% 10:00
	Для работы со строками  одними из самых важных являются функции поиска подстроки в строке find и rfind, которые имеют похожий синтаксис. 
	\texttt{
		\begin{tabbing}
			\hspace{8mm}\=\hspace{2cm}\=\kill
			s = 'Ehehe...'\\
			s.find('he') \> \> \# return 1\\
			s.rfind('he') \> \> \# return 3
		\end{tabbing}
	}
	
	При этом функция find возвращает позицию первого символа подстроки, найденной от начала строки. Но функция rfind осуществляет поиск с конца строки и, соответственно, возвращает  позицию последнего элемента подстроки.
	
	Для того чтобы найти количество вхождений подстроки в строку существует функция count. 
	В случае перекрывания подстрок, как в данном примере, 
	\texttt{
		\begin{tabbing}
			\hspace{8mm}\=\hspace{4cm}\=\kill
			s = 'AAAA'\\
			num\_AA = s.count('AA')	\> \> \# num\_AA == 2
		\end{tabbing}
	}	
	можно считать, что подстрока вошла два или три раза. В Python 3 принято считать что здесь две подстроки. 
	
	\subsection*{Литералы строк}
	
	Строка в Python записывается в одинарных или двойных кавычках. Обе записи равнозначны и существуют для того, чтобы избегать двусмысленных выражений. Впрочем, можно было бы использовать обратный слэш для экранирования кавычек (например, $\backslash$'), чтобы они воспринимались как символы внутри строки. 
	
	Если строка слишком длинная, чтобы уместиться на экране, то разумно перенести ее с помощью обратного слэша, который будет экранировать не печатающийся символ переноса каретки. При этом Python будет воспринимать ее как одну целую строку.
	
	Для того чтобы работать с многострочным объектом, его записывают в тройных кавычках (каждая из которых состоит из одной либо двух кавычек). 
	\texttt{
		\begin{tabbing}
			\hspace{8mm}\=\hspace{8mm}\=\hspace{2cm}\=\kill
			s = 'Ehehe...'\\
			s = "Ehehe..."\\
			\\
			s\_with\_comma = 'I$\backslash$'m a student'\\
			s\_with\_comma = "I'm a student"\\
			s\_with\_comma = 'Company "Parallels"'\\
			\\
			long\_s = "... \ \\
			..."\\
			\\
			many\_s = ''' ...\\
			...\\
			... '''	
		\end{tabbing}
	}
	
	Для написания строк бывает удобно пользоваться специальными символами, такими как\\
	<<$\backslash n$>> -- разрыв строки\\
	<<$\backslash t$>> -- табуляция
	
	Для того чтобы хранить в строке путь к некоторому файлу в Windows, необходимо экранировать обратный слэш (т.е. сделать его двойным). Можно также написать перед открывающей кавычкой символ \emph{r}. Тогда все, что находится внутри кавычек будет восприниматься так, как написано.
	\texttt{
		\begin{tabbing}
			\hspace{8mm}\=\hspace{8mm}\=\hspace{2cm}\=\kill
			path = 'C:$\backslash\backslash${}my$\backslash\backslash$1.py'\\
			path = r'C:$\backslash${}my$\backslash$1.py'
		\end{tabbing}
	}
	
	\subsection*{Наивный поиск подстроки в строке}
	
	В качестве примера алгоритма можно рассмотреть поиск позиции первого элемента первого вхождения подстроки <<he>> в строку <<Ehehe...>>. 
	
	Для начала определяются их длины. Затем элементы в строке пробегаются до той позиции, при которой подстрока, начинавшаяся бы с нее, еще не вышла бы за пределы строки. 
	В теле цикла проверяется совпадение подстроки с частью строки, начинающейся с текущей позиции. В случае несовпадения происходит выход из вложенного цикла. Для того чтобы выйти из внешнего цикла сразу после нахождения нужной подстроки используется логическая переменная-флаг found. В случае, если такой подстроки нет, программа возвращает $-1$.
	\texttt{
		\begin{tabbing}
			\hspace{8mm}\=\hspace{8mm}\=\hspace{8mm}\=\hspace{2cm}\=\kill
			s = 'Ehehe...'\\
			sub = 'he'\\
			len\_s = len(s)\\
			len\_sub= len(sub)\\
			\\
			pos = 0\\
			while pos + len\_sub <= len\_s:\\
			\> found = True\\
			\> i = 0\\
			\> while i < len\_sub:\\
			\> \> if s[pos + i] != sub[i]:\\
			\> \> \> found = False\\
			\> \> i += 1\\
			\> if found:\\
			\> \> break\\
			\> pos += 1\\
			else:\\
			\> pos = -1
		\end{tabbing}
	}	
	
	
	\subsection*{Списки в Python}
	
	% 33:00
	В Python список (тип list) представляет собой массив ссылок на объекты. Он изменяем: каждую ссылку можно связать с другим объектом. 
	\texttt{
		\begin{tabbing}
			\hspace{8mm}\=\hspace{8mm}\=\hspace{2.5cm}\=\kill
			A = [1, 2, 3, 4, 5]	\> \> \> \# type(A) == class<list>\\
			A[0] = 10	\> \> \> \# type(A[0]) == int\\
			 \> \> \> \# A == [10, 2, 3, 4, 5]
		\end{tabbing}
	}
	
	Можно узнать длину списка с помощью len(A), а также 
	min(A) и 
	max(A), если объекты, с которыми связаны ссылки можно сравнивать друг с другом. 
	
	Если присвоить один список другому, то тогда оба идентификатора станут указывать на один и тот же список.
	\texttt{
		\begin{tabbing}
			\hspace{8mm}\=\hspace{8mm}\=\hspace{2cm}\=\kill
			A = [1, 2, 3]\\
			B = A\\
			B[0] = 10 \> \> \> \# B == [10, 2, 3]\\ 
			\> \> \> \# A == [10, 2, 3]
		\end{tabbing}
	}
	Данная особенность связывания справедлива и при вызове функций. Так если в объявлении функции изменить сам объект, то при выходе из нее он останется измененным.
	\texttt{
		\begin{tabbing}
			\hspace{8mm}\=\hspace{8mm}\=\hspace{2cm}\=\kill
			def f(B):\\
			\> B[0] = 10\\
			f(A)\\
			\# B[0] == 10
		\end{tabbing}
	}
	Если же в объявлении функции изменяются только локальные списки, то ссылки на них теряются при выходе из функции. 
	\texttt{
		\begin{tabbing}
			\hspace{8mm}\=\hspace{8mm}\=\hspace{2cm}\=\kill
			def f(B):\\
			\> B = B[::-1]\\
			f(A)
		\end{tabbing}
	}	
	В данном примере идентификатор B связывается сначала со списком, с которым уже связан А. По этой ссылке создается новый объект -- срез, и с ним связывается идентификатор В. Из-за этого ссылка на прежний объект теряется, и при выходе из функции список А остается неизменным.
	
	Еще один пример показывает, как можно добавить элемент в конец списка и как объединить несколько списков в один, сохраняя тот же порядок элементов, что и в исходных. 
	\texttt{
		\begin{tabbing}
			\hspace{8mm}\=\hspace{8mm}\=\hspace{2cm}\=\kill
			A = [1, 2, 3]\\
			B = A\\
			B.append(4)	\> \> \> \# B == [1, 2, 3, 4]\\
			C = A + B \> \> \> \# C == [1, 2, 3, 4, 1, 2, 3, 4]\\
		\end{tabbing}
	}	
	\subsection*{Копия списка}
	
	Создание копии списка можно проводить разными способами. Так, циклически пробежавшись по все элементам списка и добавив их в новый, получится копия прежнего. Однако то же самое можно сделать короче с использованием срезов.  
	\texttt{
		\begin{tabbing}
			\hspace{8mm}\=\hspace{8mm}\=\hspace{2cm}\=\kill
			B = []\\
			for x in A:\\
			\> B.append(x)\\
			\# shorter\\
			B = A[:]
		\end{tabbing}
	}
	
	Если необходимо обезопасить список от изменения после выполнения какой-либо функции, то разумно передавать ей не сам список, а его срез.
	Т.е. вместо f(A) писать f(A[:]).

	
	\subsection*{Присваивание в срез списка}
	
	% 51:30
	Присваивание в срез с двумя параметрами никогда не приводит к ошибке. В следующем примере элементы массива, начиная с позиции 1 до элемента с позицией 3, вырезаются, а на их место вставляется другой фрагмент.
	\texttt{
		\begin{tabbing}
			\hspace{8mm}\=\hspace{8mm}\=\hspace{3cm}\=\kill
			A = [0, 1, 2, 3, 4]\\
			A[1:3] = [10, 20, 30] \> \> \> \# A == [0, 10, 20, 30, 3, 4]\\
			A[1:4] = []			\> \> \> \# A == [0, 3, 4]
		\end{tabbing}
	}	
	Присваивание же в срез с тремя параметрами может привести к ошибке, если размер присваиваемого списка не является подходящим (не укладывается в исходном списке с заданным шагом) 
	\texttt{
		\begin{tabbing}
			\hspace{8mm}\=\hspace{8mm}\=\hspace{3cm}\=\kill
			A = [0, 1, 2, 3, 4]\\
			A[1:5:2] = [10, 20]	\> \> \> \# A == [0, 10, 2, 20, 4]\\
			A[1:5:2] = [10, 20, 30]	\> \> \> \# Error!	
		\end{tabbing}
	}	
	Также имеется возможность вставки фрагмента между соседними элементами исходного списка.
	\texttt{
		\begin{tabbing}
			\hspace{8mm}\=\hspace{8mm}\=\hspace{2cm}\=\kill
			A = [1, 2, 3]\\
			A[1:1] = [5, 6]	\> \> \# A == [1, 5, 6, 2, 3]
		\end{tabbing}
	}
	Следующий пример показывает, как с помощью присваивания в срез можно создавать списки с довольно сложным порядком следования элементов.
	\texttt{
		\begin{tabbing}
			\hspace{8mm}\=\hspace{8mm}\=\hspace{3cm}\=\kill
			\# [1, 6, 2, 5, 3, 4]\\
			A = [0]*6 \> \>	\> \# A == [0, 0, 0, 0, 0, 0]\\
			A[::2] = [1, 2, 3]\\
			A[::-2] = [4, 5, 6]
		\end{tabbing}
	}	
	
	\subsection*{Обращение списка}
	
	В некоторых случаях списки бывают достаточно большими и на их обращение с помощью создания среза может не хватить памяти.  
	\texttt{
		\begin{tabbing}
			\hspace{8mm}\=\hspace{8mm}\=\hspace{2cm}\=\kill
			A = list(map(int, input().split()))\\
			i = 0\\
			while i < len(A) // 2:\\
			\> tmp = A[i]\\
			\> A[i] = A[-1-i]\\
			\> A[-1-i] = tmp
		\end{tabbing}
	}
	Данный алгоритм прост. Для его понимания достаточно заметить, что отрицательный индекс означает перебор элементов списка от последнего (-1) к первому (-n) и что исходный список обходится только до половины, так как иначе новый список был бы обращен в прежний.
	
	\subsection*{Циклический сдвиг}
	
	% 1:05:00
	Зачастую в некоторых задачах требуется осуществить циклический сдвиг элементов в списке. Так, при сдвиге влево необходимо  сначала сохранить первый элемент, чтобы потом, после работы цикла, поставить его на последнее место.
	\texttt{
		\begin{tabbing}
			\hspace{8mm}\=\hspace{8mm}\=\hspace{2.2cm}\=\kill
			A = [0, 1, 2, 3, 4]\\
			tmp = A[0]\\
			for i in range(0, len(A) - 1):\\
			\> A[i] = A[i + 1]\\
			A[-1] = tmp \> \> \> \# A == [1, 2, 3, 4, 0]
		\end{tabbing}
	}
	При наличии необходимой памяти, то же самое можно проделать с помощью срезов.
	\texttt{
		\begin{tabbing}
			\hspace{8mm}\=\hspace{8mm}\=\hspace{2cm}\=\kill
			A[:] = A[1:] + A[0:1]
		\end{tabbing}
	}
	
	\subsection*{Кортежи}
	
	В отличие от списков \emph{кортежи} являются неизменяемыми. В следующем примере показано, как объявить кортеж и как создать пустой. Важно отметить, что для устранения путаницы с переменными числового типа вводится специальная форма записи кортежа из одного элемента.
	\texttt{
		\begin{tabbing}
			\hspace{8mm}\=\hspace{3cm}\=\kill
			A = 1, 2, 3, 4, 5 \> \> \# type(A) == class<tuple>\\
			A = ()			\>	\> \# empty tuple\\
			A = 1			\> 	\> \# type(A) == int\\
			A = (1, )		\>	\> \# type(A) == class<tuple>
		\end{tabbing}
	}	
	
\end{document}