\documentclass[a4paper,12pt]{article}
\usepackage{amsmath,amssymb,amsfonts,amsthm}
\usepackage{tikz}
\usepackage [utf8] {inputenc}
\usepackage [T2A] {fontenc} 
\usepackage[russian]{babel}
\usepackage{cmap}
\usepackage{listings}
\usepackage{color}


\definecolor{mygreen}{rgb}{0,0.6,0}
\definecolor{mygray}{rgb}{0.5,0.5,0.5}
\definecolor{mymauve}{rgb}{0.58,0,0.82} 

\lstdefinestyle{customc}{
	belowcaptionskip=1\baselineskip,
	breaklines=true,
	frame=L,
	xleftmargin=\parindent,
	language=Python,
	showstringspaces=false,
	basicstyle=\footnotesize\ttfamily,
	keywordstyle=\bfseries\color{green!40!black},
	commentstyle=\itshape\color{purple!40!black},
	identifierstyle=\color{blue},
	stringstyle=\color{orange},
}


\newtheorem{defenition}{Определение}
\newtheorem{theorem}{Теорема}
\newtheorem{problem}{Задача}

\lstset{escapechar=@,style=customc}

% Так ссылки в PDF будут активны
\usepackage[unicode]{hyperref}

% вы сможете вставлять картинки командой \includegraphics[width=0.7\textwidth]{ИМЯ ФАЙЛА}
% получается подключать, как минимум, файлы .pdf, .jpg, .png.
\usepackage{graphicx}
% Если вы хотите явно указать поля:
\usepackage[margin=1in]{geometry}
% Или если вы хотите задать поля менее явно (чем больше DIV, тем больше места под текст):
% \usepackage[DIV=10]{typearea}

\usepackage{fancyhdr}

\newcommand{\bbR}{\mathbb R}%теперь вместо длинной команды \mathbb R (множество вещественных чисел) можно писать короткую запись \bbR. Вместо \bbR вы можете вписать любую строчку букв, которая начинается с '\'.
\newcommand{\eps}{\varepsilon}
\newcommand{\bbN}{\mathbb N}
\newcommand{\dif}{\mathrm{d}}

\pagestyle{fancy}
\makeatletter % сделать "@" "буквой", а не "спецсимволом" - можно использовать "служебные" команды, содержащие @ в названии
\fancyhead[L]{\footnotesize Информатика\\ Внимание: конспект не проверялся преподавателями —- всегда используйте рекомендуемую литературу при подготовке к экзамену!}%Это будет написано вверху страницы слева
\fancyhead[R]{}
\fancyfoot[L]{}%имя автора будет написано внизу страницы слева
\fancyfoot[R]{\thepage}%номер страницы —- внизу справа
\fancyfoot[C]{ Замечания и предложения просьба писать на \href{mailto:pulsar@phystech.edu}{pulsar@phystech.edu}}%по центру внизу страницы пусто

\renewcommand{\maketitle}{%
	\noindent{\bfseries\scshape\large\@title\ \mdseries\upshape}\par
	\noindent {\large\itshape\@author}
	\vskip 2ex}
\makeatother
\def\dd#1#2{\frac{\partial#1}{\partial#2}}
\def\ddd#1#2#3{\frac{\partial^2#1}{\partial#2\partial#3}}

\title{Основы компьютерных сетей. II.}
\author{Тимофей Хирьянов}
\date{26 апреля 2016 г.}

\begin{document}
	\maketitle
	\section{Уровни организации модели OSI ISO.}
		\subsection{Пользовательский уровень (Application Layer)}
		Пользовательский уровень — верхний уровень модели, обеспечивающий взаимодействие пользовательских приложений с сетью.
		Примеры:RDP, HTTP, SMTP, SNMP, POP3, FTP, XMPP, OSCAR, Modbus, SIP, TELNET
		\subsection{Представительский уровень (Presentation Layer)}	
		Представительский уровень обеспечивает преобразование протоколов и кодирование/декодирование данных. Запросы приложений, полученные с прикладного уровня, на уровне представления преобразуются в формат для передачи по сети, а полученные из сети данные преобразуются в формат приложений. На этом уровне может осуществляться сжатие/распаковка или шифрование/дешифрование, а также перенаправление запросов другому сетевому ресурсу, если они не могут быть обработаны локально. 
		
		Представительский уровень имеет дело не только с форматами и представлением данных, он также занимается структурами данных, которые используются программами. Таким образом, этот уровень обеспечивает организацию данных при их пересылке.
		
		Протоколы уровня представления: AFP — Apple Filing Protocol, ICA — Independent Computing Architecture, LPP — Lightweight Presentation Protocol, NCP — NetWare Core Protocol, NDR — Network Data Representation, XDR — eXternal Data Representation, X.25 PAD — Packet Assembler/Disassembler Protocol.
		
		\subsection{Сеансовый уровень (session layer)}
		Сеансовый уровень модели обеспечивает поддержание сеанса связи, позволяя приложениям взаимодействовать между собой длительное время. Уровень управляет созданием/завершением сеанса, обменом информацией, синхронизацией задач, определением права на передачу данных и поддержанием сеанса в периоды неактивности приложений.
		
		Протоколы сеансового уровня: ADSP (AppleTalk Data Stream Protocol), ASP (AppleTalk Session Protocol), H.245 (Call Control Protocol for Multimedia Communication), ISO-SP (OSI Session Layer Protocol (X.225, ISO 8327)), iSNS (Internet Storage Name Service), L2F (Layer 2 Forwarding Protocol), L2TP (Layer 2 Tunneling Protocol), NetBIOS (Network Basic Input Output System), PAP (Password Authentication Protocol), PPTP (Point-to-Point Tunneling Protocol), RPC (Remote Procedure Call Protocol), RTCP (Real-time Transport Control Protocol), SMPP (Short Message Peer-to-Peer), SCP (Session Control Protocol), ZIP (Zone Information Protocol), SDP (Sockets Direct Protocol).
	
		\subsection{Транспортный уровень (transport layer)}
		Транспортный уровень модели предназначен для обеспечения надёжной передачи данных от отправителя к получателю. При этом уровень надёжности может варьироваться в широких пределах. Существует множество классов протоколов транспортного уровня, начиная от протоколов, предоставляющих только основные транспортные функции (например, функции передачи данных без подтверждения приема), и заканчивая протоколами, которые гарантируют доставку в пункт назначения нескольких пакетов данных в надлежащей последовательности, мультиплексируют несколько потоков данных, обеспечивают механизм управления потоками данных и гарантируют достоверность принятых данных. Например, UDP ограничивается контролем целостности данных в рамках одной датаграммы, и не исключает возможности потери пакета целиком, или дублирования пакетов, нарушение порядка получения пакетов данных; TCP обеспечивает надёжную непрерывную передачу данных, исключающую потерю данных или нарушение порядка их поступления или дублирования, может перераспределять данные, разбивая большие порции данных на фрагменты и наоборот склеивая фрагменты в один пакет.
		
		Протоколы транспортного уровня: ATP (AppleTalk Transaction Protocol), CUDP (Cyclic UDP), DCCP (Datagram Congestion Control Protocol), FCP (Fiber Channel Protocol), IL (IL Protocol), NBF (NetBIOS Frames protocol), NCP (NetWare Core Protocol), SCTP (Stream Control Transmission Protocol), SPX (Sequenced Packet Exchange), SST (Structured Stream Transport), TCP (Transmission Control Protocol), UDP (User Datagram Protocol).
		\subsection{Сетевой уровень (Network Layer)}
		Сетевой уровень модели предназначен для определения пути передачи данных. Отвечает за трансляцию логических адресов и имён в физические, определение кратчайших маршрутов, коммутацию и маршрутизацию, отслеживание неполадок и «заторов» в сети.
		
		Протоколы сетевого уровня маршрутизируют данные от источника к получателю. Работающие на этом уровне устройства (маршрутизаторы) условно называют устройствами третьего уровня (по номеру уровня в модели OSI).
		
		Протоколы сетевого уровня: IP/IPv4/IPv6 (Internet Protocol), IPX (Internetwork Packet Exchange, протокол межсетевого обмена), X.25 (частично этот протокол реализован на уровне 2), CLNP (сетевой протокол без организации соединений), IPsec (Internet Protocol Security). Протоколы маршрутизации - RIP (Routing Information Protocol), OSPF (Open Shortest Path First).
		\subsection{Канальный уровень (Data Link layer)}
		Канальный уровень предназначен для обеспечения взаимодействия сетей на физическом уровне и контроля за ошибками, которые могут возникнуть. Полученные с физического уровня данные, представленные в битах, он упаковывает в кадры, проверяет их на целостность и, если нужно, исправляет ошибки (формирует повторный запрос поврежденного кадра) и отправляет на сетевой уровень. Канальный уровень может взаимодействовать с одним или несколькими физическими уровнями, контролируя и управляя этим взаимодействием.
		
		Спецификация IEEE 802 разделяет этот уровень на два подуровня: MAC (англ. media access control) регулирует доступ к разделяемой физической среде, LLC (англ. logical link control) обеспечивает обслуживание сетевого уровня.
		
		На этом уровне работают коммутаторы, мосты и другие устройства. Говорят, что эти устройства используют адресацию второго уровня (по номеру уровня в модели OSI).
		
		Протоколы канального уровня: ARCnet, ATM, Controller Area Network (CAN), Econet, IEEE 802.3 (Ethernet), Ethernet Automatic Protection Switching (EAPS), Fiber Distributed Data Interface (FDDI), Frame Relay, High-Level Data Link Control (HDLC), IEEE 802.2 (provides LLC functions to IEEE 802 MAC layers), Link Access Procedures, D channel (LAPD), IEEE 802.11 wireless LAN, LocalTalk, Multiprotocol Label Switching (MPLS), Point-to-Point Protocol (PPP), Point-to-Point Protocol over Ethernet (PPPoE), Serial Line Internet Protocol (SLIP, устарел), StarLan, Token ring, Unidirectional Link Detection (UDLD), x.25, ARP.
		\subsection{Физический уровень (physical layer)}
		Физический уровень (англ. physical layer) — нижний уровень модели, который определяет метод передачи данных, представленных в двоичном виде, от одного устройства (компьютера) к другому. Составлением таких методов занимаются разные организации, в том числе: Институт инженеров по электротехнике и электронике, Альянс электронной промышленности, Европейский институт телекоммуникационных стандартов и другие. Осуществляют передачу электрических или оптических сигналов в кабель или в радиоэфир и, соответственно, их приём и преобразование в биты данных в соответствии с методами кодирования цифровых сигналов.
		
		На этом уровне также работают концентраторы, повторители сигнала и медиаконвертеры.
		
		Функции физического уровня реализуются на всех устройствах, подключенных к сети. Со стороны компьютера функции физического уровня выполняются сетевым адаптером или последовательным портом. К физическому уровню относятся физические, электрические и механические интерфейсы между двумя системами. Физический уровень определяет такие виды сред передачи данных как оптоволокно, витая пара, коаксиальный кабель, спутниковый канал передач данных и т. п. Стандартными типами сетевых интерфейсов, относящимися к физическому уровню, являются: V.35, RS-232, RS-485, RJ-11, RJ-45, разъемы AUI и BNC.
		
		Протоколы физического уровня: IEEE 802.15 (Bluetooth), IRDA, EIA RS-232, EIA-422, EIA-423, RS-449, RS-485, DSL, ISDN, SONET/SDH, 802.11 Wi-Fi, Etherloop, GSM Um radio interface, ITU и ITU-T, TransferJet, ARINC 818, G.hn/G.9960.
		\section{DHCP-сервер}
		DHCP (англ. Dynamic Host Configuration Protocol — протокол динамической настройки узла) — сетевой протокол, позволяющий компьютерам автоматически получать IP-адрес и другие параметры, необходимые для работы в сети TCP/IP. Данный протокол работает по модели «клиент-сервер». Для автоматической конфигурации компьютер-клиент на этапе конфигурации сетевого устройства обращается к так называемому серверу DHCP и получает от него нужные параметры. Сетевой администратор может задать диапазон адресов, распределяемых сервером среди компьютеров. Это позволяет избежать ручной настройки компьютеров сети и уменьшает количество ошибок. Протокол DHCP используется в большинстве сетей TCP/IP.
		Протокол DHCP предоставляет три способа распределения IP-адресов:
		\begin{enumerate}
		\item Ручное распределение. При этом способе сетевой администратор сопоставляет аппаратному адресу (для Ethernet сетей это MAC-адрес) каждого клиентского компьютера определённый IP-адрес. Фактически, данный способ распределения адресов отличается от ручной настройки каждого компьютера лишь тем, что сведения об адресах хранятся централизованно (на сервере DHCP), и потому их проще изменять при необходимости.\\
		\item Автоматическое распределение. При данном способе каждому компьютеру на постоянное использование выделяется произвольный свободный IP-адрес из определённого администратором диапазона. \\
		\item Динамическое распределение. Этот способ аналогичен автоматическому распределению, за исключением того, что адрес выдаётся компьютеру не на постоянное пользование, а на определённый срок. Это называется арендой адреса. По истечении срока аренды IP-адрес вновь считается свободным, и клиент обязан запросить новый (он, впрочем, может оказаться тем же самым). Кроме того, клиент сам может отказаться от полученного адреса.
	\end{enumerate}
		
\end{document}
