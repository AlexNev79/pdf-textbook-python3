\documentclass[a4paper,12pt]{article}

%%% Работа с русским языком % для pdfLatex
\usepackage{cmap}					% поиск в~PDF
\usepackage{mathtext} 				% русские буквы в~фомулах
\usepackage[T2A]{fontenc}			% кодировка
\usepackage[utf8]{inputenc}			% кодировка исходного текста
\usepackage[english,russian]{babel}	% локализация и переносы
\usepackage{indentfirst} 			% отступ 1 абзаца

%%% Работа с русским языком % для XeLatex
%\usepackage[english,russian]{babel}   %% загружает пакет многоязыковой вёрстки
%\usepackage{fontspec}      %% подготавливает загрузку шрифтов Open Type, True Type и др.
%\defaultfontfeatures{Ligatures={TeX},Renderer=Basic}  %% свойства шрифтов по умолчанию
%\setmainfont[Ligatures={TeX,Historic}]{Times New Roman} %% задаёт основной шрифт документа
%\setsansfont{Comic Sans MS}                    %% задаёт шрифт без засечек
%\setmonofont{Courier New}
%\usepackage{indentfirst}
%\frenchspacing

%%% Дополнительная работа с математикой
\usepackage{amsfonts,amssymb,amsthm,mathtools}
\usepackage{amsmath}
\usepackage{icomma} % "Умная" запятая: $0,2$ --- число, $0, 2$ --- перечисление
\usepackage{upgreek}

%% Номера формул
%\mathtoolsset{showonlyrefs=true} % Показывать номера только у тех формул, на которые есть \eqref{} в~тексте.

%%% Страница
\usepackage{extsizes} % Возможность сделать 14-й шрифт

%% Шрифты
\usepackage{euscript}	 % Шрифт Евклид
\usepackage{mathrsfs} % Красивый матшрифт

%% Свои команды
\DeclareMathOperator{\sgn}{\mathop{sgn}} % создание новой конанды \sgn (типо как \sin)
\usepackage{csquotes} % ещё одна штука для цитат
\newcommand{\pd}[2]{\ensuremath{\cfrac{\partial #1}{\partial #2}}} % частная производная
\newcommand{\abs}[1]{\ensuremath{\left|#1\right|}} % модуль
\renewcommand{\phi}{\ensuremath{\varphi}} % греческая фи
\newcommand{\pogk}[1]{\!\left(\cfrac{\sigma_{#1}}{#1}\right)^{\!\!\!2}\!} % для погрешностей

% Ссылки
\usepackage{color} % подключить пакет color
% выбрать цвета
\definecolor{BlueGreen}{RGB}{49,152,255}
\definecolor{Violet}{RGB}{120,80,120}
% назначить цвета при подключении hyperref
\usepackage[unicode, colorlinks, urlcolor=blue, linkcolor=blue, pagecolor=blue, citecolor=blue]{hyperref} %синие ссылки
%\usepackage[unicode, colorlinks, urlcolor=black, linkcolor=black, pagecolor=black, citecolor=black]{hyperref} % для печати (отключить верхний!)


%% Перенос знаков в~формулах (по Львовскому)
\newcommand*{\hm}[1]{#1\nobreak\discretionary{}
	{\hbox{$\mathsurround=0pt #1$}}{}}

%%% Работа с картинками
\usepackage{graphicx}  % Для вставки рисунков
\graphicspath{{images/}{images2/}}  % папки с картинками
\setlength\fboxsep{3pt} % Отступ рамки \fbox{} от рисунка
\setlength\fboxrule{1pt} % Толщина линий рамки \fbox{}
\usepackage{wrapfig} % Обтекание рисунков и таблиц текстом
\usepackage{multicol}

%%% Работа с таблицами
\usepackage{array,tabularx,tabulary,booktabs} % Дополнительная работа с таблицами
\usepackage{longtable}  % Длинные таблицы
\usepackage{multirow} % Слияние строк в~таблице
\usepackage{caption}
\captionsetup{labelsep=period, labelfont=bf}

%%% Оформление
\usepackage{indentfirst} % Красная строка
%\setlength{\parskip}{0.3cm} % отступы между абзацами
%%% Название разделов
\usepackage{titlesec}
\titlelabel{\thetitle.\quad}
\renewcommand{\figurename}{\textbf{Рис.}}		%Чтобы вместо figure под рисунками писал "рис"
\renewcommand{\tablename}{\textbf{Таблица}}		%Чтобы вместо table над таблицами писал Таблица

%%% Теоремы
\theoremstyle{plain} % Это стиль по умолчанию, его можно не переопределять.
\newtheorem{theorem}{Теорема}[section]
\newtheorem{proposition}[theorem]{Утверждение}

\theoremstyle{definition} % "Определение"
\newtheorem{definition}{Определение}[section]
\newtheorem{corollary}{Следствие}[theorem]
\newtheorem{problem}{Задача}[section]

\theoremstyle{remark} % "Примечание"
\newtheorem*{nonum}{Решение}
\newtheorem{zamech}{Замечание}[theorem]

%%% Правильные мат. символы для русского языка
\renewcommand{\epsilon}{\ensuremath{\varepsilon}}
\renewcommand{\phi}{\ensuremath{\varphi}}
\renewcommand{\kappa}{\ensuremath{\varkappa}}
\renewcommand{\le}{\ensuremath{\leqslant}}
\renewcommand{\leq}{\ensuremath{\leqslant}}
\renewcommand{\ge}{\ensuremath{\geqslant}}
\renewcommand{\geq}{\ensuremath{\geqslant}}
\renewcommand{\emptyset}{\varnothing}

%%% Для лекций по инфе
\usepackage{alltt}
\newcounter{infa}[section]
\newcounter{num}
\definecolor{infa}{rgb}{0, 0.2, 0.89}
\definecolor{infa1}{rgb}{0, 0.3, 1}
\definecolor{grey}{rgb}{0.5, 0.5, 0.5}
\newcommand{\tab}{\ \ \ }
\newcommand{\com}[1]{{\color{grey}\##1}}
\newcommand{\num}{\addtocounter{num}{1}\arabic{num}\tab}
\newcommand{\defi}{{\color{infa}def}}
\newcommand{\globali}{{\color{infa}global}}
\newcommand{\ini}{{\color{infa}in}}
\newcommand{\rangei}{{\color{infa}range}}
\newcommand{\fori}{{\color{infa}for}}
\newcommand{\ifi}{{\color{infa}if}}
\newcommand{\elsei}{{\color{infa}else}}
\newcommand{\printi}{{\color{infa1}print}}
\newcommand{\enumeratei}{{\color{infa1}enumerate}}
\newcommand{\maxi}{{\color{infa}max}}
\newcommand{\classi}{{\color{infa}class}}
\newcommand{\returni}{{\color{infa}return}}
\newcommand{\elifi}{{\color{infa}elif}}
\newenvironment{infa}[1]{
	
	\vspace{0.5cm}
	\addtocounter{infa}{1}%
	\noindent{\large \textbf{Программа №\thesection.\arabic{infa}.}\ \textbf{#1}}%
	\begin{alltt}%
	}{\end{alltt}
	\setcounter{num}{0}
	\vspace{0.1cm}}
%Пример кода:
%\begin{infa}{Поразрядная сортировка}
%	\ \num \defi count_sort(a):\tab \com{определяет нашу функцию}
%	\ \num \tab m = \maxi(a)+1
%	\ \num \tab q = [0]*m
%	\ \num \tab \fori x \ini a:
%	\ \num \tab \tab q[x] += 1
%	\ \num \tab pos = 0
%	\ \num \tab \fori x \ini q:
%	\ \num \tab \tab \fori i \ini \rangei(q[x]):
%	\ \num \tab \tab \tab a[pos] = x
%	\num \tab \tab \tab pos += 1
%\end{infa}

\usepackage[left=1.27cm,right=1.27cm,top=1.27cm,bottom=2cm]{geometry}
%\hbox to\textwidth{команда колонтитула}
\begin{document}
\newcounter{lec}
\newcommand{\lec}[1]{\addtocounter{lec}{1} \setcounter{section}{0}%
\begin{center}
{\LARGE ЛЕКЦИЯ \arabic{lec}%
\vspace{2mm}%

\textbf{#1}%
}
\end{center}
}
\newpage
\
\lec{Повторение}
\section{Ссылочная модель данных}
Ссылочная модель данных - объекты существуют независимо от времен.
\begin{alltt}
>>> 2+3
5
\end{alltt}
Создаются два объекта типа int.
\begin{alltt}
2.__add__(3) - метод добавления
\end{alltt}
\begin{alltt}
	x = 2+3 # x ссылается на объект 5
	x = 'Hello' # возникает объект строки
\end{alltt}
После вывода результата объекты 2, 3, 5 удаляются, т.к. нет ссылки на эти объекты.
Если x перестал ссылаться на 5, объект 5 уничтожается.
\begin{alltt}
x = 5
y = x
x = 'Hello'
y = None
\end{alltt}
После того, как y начал ссылаться на None, объект 5 уничтожается.
\begin{alltt}
	x = (2+3+5)**5**5 # приоритет у возведения в степени
\end{alltt}

Т.е. в начале выполняется 5**5, а потом (2+3+5)**25.

Начиная с версии Python 3.6, можно разделять числа так:
\begin{alltt}
x = 100_001
y = 0xAF_DC
\end{alltt}
\section{Пространство имен}
4 пространства: локальные, окружающие, глобальные, строенные (LEGB).

y = 10*x+7
Сначала будет происходить поиск локального x, затем в надпространстве, затем в глобальных, в последнюю очередь - во встроенных переменных.

Если написать 
max = 10,
то функция max перестанет работать.

Пример:
\begin{infa}{}
\num \defi f(A):
\num \tab A = A + 10 \com{если написать A += 10 ошибки не будет}
\num B = [1, 2, 3]
\num f(B)
\num \printi(*B) \com{1 2 3}
\end{infa}

Строки и числа --- неизменяемые объекты.
f является именем объекта functional.

def --- по сути это операция создания нового объекта.

Любой вызов функции порождает свое пространство имен, которое перестанет существовать после выполнения return. 

A начинает ссылаться на [1, 2, 3]. После конкатинации A начинает ссылаться на [1, 2, 3, 4]. A ссылается на глобальный объект, и начинает ссылаться на локальный объект. После return уничтожается A и список [1, 2, 3, 4].

Или можно записать так:
\begin{infa}{}
	\num \defi f(A):
	\num \tab A.append(10)
	\num B = [1, 2, 3]
	\num f(B)
	\num \printi(*B) \com{1 2 3}
\end{infa}

Функция должна что-то возвращать. В частном случае, можно возвращать несколько параметров. Нарушить ссылочную модель можно только "залезая"\ в глобальные имена.

\begin{infa}{}
	\num d=1
	\num \defi f(A):
	\num \tab \globali d
	\num \tab A.append(d)
	\num \tab d = d + 1
\end{infa}







\section{ООП}
Тип тоже является объектом типа тип. 
\begin{infa}{}
\num \classi Base:
\num \tab x = 10
\num \tab \defi g(self, x0):
\num \tab \tab self.x = x0
\num b = Base()
\num \printi(b.x)
\end{infa}

\begin{infa}{}
	\num \classi Base:
	\num \tab x = 10
	\num \tab \defi g(self, x0)
	\num \tab \tab self.x += x0
	\num b = Base()
	\num \printi(b.x)
\end{infa}

Нужно создавать атрибуты только в методе init!

\section{Элементы функционального программирования}

\subsection{Функция map}
\begin{alltt}
x, y, z = map(int, input().split()) \com{считывание трех чисел с клавиатуры}
\end{alltt}

Функция map применяет к каждому объекту функцию, которую мы написали. В нее же можно написать свою функцию:
\begin{alltt}
x, y, z = map(lambda x: int(x)**2, input().split())
	A == list(map(int, range(100)))
	B = map(float, int(x) for x in input().split())
\end{alltt}

map возвращает объект типа map.

\subsection{Применение lambda-функций}
$x^2+e^{1/x}+\ln x$
\begin{alltt}
x = decompoused_value # нужно объявлять функцию
(lambda x: x**2 + exp(1/x)+ln(x))(2) #хороший пример использования lambda
\end{alltt}

\subsection{Функция enumerate}
\begin{infa}{}
\num A = [10, 20, 30]
\num \fori i,x \ini \enumeratei(A): \com{А используется как итерируемый объект}
\num \tab \printi(i,x) \com{(0,10), (1,20), (2,30) - эту конструкцию нам вернет enumerate(A)}
\num \tab x = x + 1	\com{Значение в массиве не изменилось, мы только испортили х}
\end{infa}
\subsection{Функция zip}

\begin{alltt}
A = [1, 2, 3, 4, 5]
B = 'Hello'
c = list(zip(A,B)) \com{результат есть zip-object}
\end{alltt} 







\end{document} 