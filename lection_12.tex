% !TeX spellcheck = russian_english
% !TeX encoding = UTF-8
\documentclass[a4paper, fleqn]{article}

\usepackage{indentfirst} % Красная строка
\usepackage[T2A]{fontenc} % Поддержка русских букв
\usepackage[utf8]{inputenc} % Кодировка utf8
\usepackage[russian]{babel} % руссификация
\usepackage{amssymb} % дополнительные символы
\usepackage{textcomp} % дополнительные текстовые символы
\usepackage{amsmath} % матрицы
%\usepackage{pdfpages}
%\usepackage{pgfplots}
%\usepackage{pgfplotstable}
\usepackage{listings}
% листинги
\lstset{language=Python, tabsize=4, language=Python}


\textwidth = 16 cm
\oddsidemargin= 0 cm
\evensidemargin= 1 cm


% \abovedisplayskip = 0 pt %.2\abovedisplayskip
% \belowdisplayskip = .2\belowdisplayskip
% \abovedisplayshortskip=.2\abovedisplayshortskip
% \belowdisplayshortskip=.2\belowdisplayshortskip
% \topsep= 0 pt

% \setlength{\mathindent}{1.2 cm}
% \setlength{\topsep}{0 pt}
% \setlength{\abovedisplayskip}{0 pt}

% \clubpenalty = 5000 % запрет висячих строк
% \widowpenalty = 5000
\binoppenalty=10000
\relpenalty=10000

% собственные команды и окружения

\newenvironment{example}[1][]{\medskip \noindent \textbf{Пример. #1}\par \nopagebreak}{\medskip \par} % окружение-"пример"


\title{Лекция \textnumero\,12}
% {\huge \vspace{3 cm}}}

\author{Т.\,Ф. Хирьянов}

\date{}

\begin{document}
	\maketitle

\subsection*{Объектно-ориентированное программирование}
\texttt{
	\begin{tabbing}
		\hspace{8mm}\=\hspace{8mm}\=\hspace{8mm}\=\hspace{3cm}\=\kill
		class Rabbit:\\
		\> size = None\\
		\> sytost = 0\\
		\> def set\_size(self, size):\\
		\> \> self.size = size\\
		\> def get\_size(self):\\
		\> \> return self.size\\
		\> def feed(self, meal):\\
		\> \> self.sytost += calorii(meal)
	\end{tabbing}
}


\texttt{
	\begin{tabbing}
		\hspace{8mm}\=\=\hspace{3cm}\=\kill
		a = Rabbit()\\
		b = Rabbit()\\
		a.feed(carrot) \> \# Rabbit.feed(a, carrot)
	\end{tabbing}
}

\subsection*{Полиморфизм в Python}

\texttt{
	\begin{tabbing}
		\hspace{8mm}\=\hspace{8mm}\=\hspace{8mm}\=\hspace{3cm}\=\kill
		def f(x):\\
		\> return x*x\\
		def f(x, y):\\
		\> return x*y
	\end{tabbing}
}

\texttt{
	\begin{tabbing}
		\hspace{8mm}\=\hspace{8mm}\=\hspace{8mm}\=\hspace{3cm}\=\kill
		def f(x, y=None):\\
		\> if y == None:\\
		\> \> y = x\\
		\> return x*y
	\end{tabbing}
}
	
	\subsection*{setter и getter}
	
	\texttt{
		\begin{tabbing}
			\hspace{8mm}\=\hspace{8mm}\=\hspace{8mm}\=\hspace{3cm}\=\kill
			class Student:\\
			\> def \_\_init\_\_(self, age, name):\> \> \# Конструктор\\
			\> \> self.\_age = age\\
			\> \> self.\_name = name\\
			\> def aging(self):\\
			\> \> self.\_age += 1\\
			\> \> print('Ура! Мне', self.\_age, 'лет!')\\
			\> @property\\
			def \_age(self): \> \> \# <<getter>> (свойство)\\
			\> return self.age\\
			\> @age.setter\\
			\> def age(self, age):\\
			\> \> self.age = age
		\end{tabbing}
	}
	
	\texttt{
		\begin{tabbing}
			\hspace{8mm}\=\hspace{8mm}\=\hspace{8mm}\=\hspace{3cm}\=\kill
			a = Student(17, 'Вася')\\
			a.age = 18 \> \> \# Нельзя!\\
			x = a.age
		\end{tabbing}
	}
	
	\subsection*{Определение функций}
	
	\texttt{
		\begin{tabbing}
			\hspace{8mm}\=\hspace{8mm}\=\hspace{8mm}\=\hspace{3cm}\=\kill
			def productor(x):\\
			\> return x*productor.p\\
			productor.p = 5\\
			print(productor(2)) \> \> \# 10
		\end{tabbing}
	}
	
	\texttt{
		\begin{tabbing}
			\hspace{8mm}\=\hspace{8mm}\=\hspace{8mm}\=\hspace{3cm}\=\kill
			class Vector2D:\\
			\> def \_\_init\_\_(self.x=0, y=0):\\
			\> \> self.\_x = x\\
			\> \> self.\_y = y\\
			\> def \_\_add\_\_(self, other):\\
			\> \> x = self.\_x + other.\_x\\
			\> \> y = self.\_y + other.\_y\\
			\> \> return Vector2D(x, y)\\
			\> def \_\_mul\_\_(self, other):\\
			\> \> if isInstance(other, Vector2D):\\
			\> \> \> return self.\_x *other.\_x + self.\_y *other.\_y\\
			\> \> else:\\
			\> \> \> return Vector2D(self.\_x * other, self.\_y * other)
		\end{tabbing}
	}
	
	\texttt{
		\begin{tabbing}
			\hspace{8mm}\=\hspace{8mm}\=\hspace{8mm}\=\hspace{3cm}\=\kill
			def geom\_range(start, stop, d):\\
			\> x = start\\
			\> while x < stop:\\
			\> \> yield x\\
			\> \> x *= d
		\end{tabbing}
	}
	
	\texttt{
		\begin{tabbing}
			\hspace{8mm}\=\hspace{8mm}\=\hspace{8mm}\=\hspace{3cm}\=\kill
			for x in geom\_range(1, 1025, 2):\\
			\> print(x)
		\end{tabbing}
	}
	
	\texttt{
		\begin{tabbing}
			\hspace{8mm}\=\hspace{8mm}\=\hspace{8mm}\=\hspace{3cm}\=\kill
			g\_num = dict(zip(range(0, 11), geom\_range(1, 1025, 2))\\
			x, y, z = list(map(int, input().split())\\
			x, y, z = [int(x) for x in input().split()]
		\end{tabbing}
	}
	
	\texttt{
		\begin{tabbing}
			\hspace{8mm}\=\hspace{8mm}\=\hspace{8mm}\=\hspace{3cm}\=\kill
			a = [1]\\
			a.append(a) \> \> \# [1, *]\\
			a.append(a)\\
			x = a\\
			for i in range(10):\\
			\> print(x[0])\\
			\> x = x[1]\\
		\end{tabbing}
	}
\end{document}