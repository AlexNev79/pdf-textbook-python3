% !TeX spellcheck = russian_english
% !TeX encoding = UTF-8
\documentclass[a4paper, fleqn]{article}

\usepackage{indentfirst} % Красная строка
\usepackage[T2A]{fontenc} % Поддержка русских букв
\usepackage[utf8]{inputenc} % Кодировка utf8
\usepackage[russian]{babel} % руссификация
\usepackage{amssymb} % дополнительные символы
\usepackage{textcomp} % дополнительные текстовые символы
\usepackage{amsmath} % матрицы
%\usepackage{pdfpages}
%\usepackage{pgfplots}
%\usepackage{pgfplotstable}
\usepackage{listings}
% листинги
\lstset{language=Python, tabsize=4, language=Python}


\textwidth = 16 cm
\oddsidemargin= 0 cm
\evensidemargin= 1 cm


% \abovedisplayskip = 0 pt %.2\abovedisplayskip
% \belowdisplayskip = .2\belowdisplayskip
% \abovedisplayshortskip=.2\abovedisplayshortskip
% \belowdisplayshortskip=.2\belowdisplayshortskip
% \topsep= 0 pt

% \setlength{\mathindent}{1.2 cm}
% \setlength{\topsep}{0 pt}
% \setlength{\abovedisplayskip}{0 pt}

% \clubpenalty = 5000 % запрет висячих строк
% \widowpenalty = 5000
\binoppenalty=10000
\relpenalty=10000

% собственные команды и окружения

\newenvironment{example}[1][]{\medskip \noindent \textbf{Пример. #1}\par \nopagebreak}{\medskip \par} % окружение-"пример"


\title{Лекция \textnumero\,6}
% {\huge \vspace{3 cm}}}

\author{Т.\,Ф. Хирьянов}

\date{}

\begin{document}
	\maketitle
	
	\section*{Структурное программирование}
	
	
	\subsection*{Базовые принципы структурного программирования}
	
	% 09:25
	\begin{itemize}
		\item Программа состоит из 
		\begin{enumerate}
			\item \emph{последовательного исполнения}
			\item \emph{ветвлений}
			\item \emph{циклов}
		\end{enumerate}
		\item Повторяющиеся участки кода оформляют в виде \emph{функций}
		\item Разработка программы осуществляется пошагово \emph{сверху-вниз}
	\end{itemize}
	
	В качестве иллюстрации можно рассмотреть следующую задачу. На вход поступают строки с автомобильным номерам и величиной скорости, с которой транспортное средство проезжает мимо поста ГИБДД. 
	
	\begin{lstlisting}
		\[A238BE	73	\longrightarrow 100\]
		\[B202CC	84	\longrightarrow	200\]
		\[B555PH	71	\longrightarrow	1000\]
		\[...\]
		\[A999AA	100\]	
	\end{lstlisting}
	
	\begin{lstlisting}
		def solve_task():
			print(count_salary())
			
		def count_salary():
			salary = 0
			licence_num, speed = input().split()
			while not chief(licence_num):
				if float(speed) > 60:
					salary += count_tax(licence_num)
				licence_num, speed = input().split()
			return salary
			
		def chief(licence_num):
			return True		# FIXME
			
		def count_tax(licence_num):
			return 0		# FIXME
				
	\end{lstlisting}
	
	% 38:00 рисунок
	
	% 41:00
	\begin{lstlisting}
		def count_tax(licence_num):	
			""" 2 numbers - 200
				3 numbers - 1000
				else - 100 """	
			pass
	\end{lstlisting}
	
	% 56:00
	\begin{lstlisting}
		def my_print(s, sep='.'):
			res = ''
			for symbol in s:
				res += symbol + sep
			print(res)
			
		my_print('Hello')
	\end{lstlisting}
	
	\begin{lstlisting}
		def f(x, y):
			return x/y
			
		f(1, 2)		# f(x=1, y=2)
	\end{lstlisting}
	
	\begin{lstlisting}
		m = 0
		for x in A:
			if x > m:
				m = x
	\end{lstlisting}
	
	\begin{lstlisting}
		for symbol in 'Hellow':
			print(symbol)
	\end{lstlisting}
	
	\begin{lstlisting}
		A = [int(x) for x in input().split()]
		B = [x**2 for x in A]
		C = [x for x in A if x % 2 == 0]
	\end{lstlisting}
	
	\begin{lstlisting}
		A = []
		A.append([1, 2, 3])
		A.append([4, 5, 6])
		
		A[1][2] == 7
	\end{lstlisting}
	
	\begin{lstlisting}
		N = int(input())
		A = [0]*N
		A = [[0]*N]*M
	\end{lstlisting}
	
	\begin{lstlisting}
		A = [[0]*N for i in range(M)]
		A = [[i*j for i in range(10)]\
			for j in range(10)]
	\end{lstlisting}
	
	\begin{lstlisting}
		A = [[0]*N for i in range(M)]
		A = [[4*j + i for i in range(4)]\
			for j in range(3)]
	\end{lstlisting}
	
	\begin{tabular}{|c|c|c|c|}
		\hline \rule[-2ex]{0pt}{5.5ex}  0&  1&  2&  3\\ 
		\hline \rule[-2ex]{0pt}{5.5ex}  4&  5&  6&  7\\ 
		\hline \rule[-2ex]{0pt}{5.5ex}  8&  9&  10&  11\\ 
		\hline 
	\end{tabular}
	
	\begin{lstlisting}
		def plus(a, b):
			return a + b
		
		plus (1, 2)
		plus (1.5, 7.5)
		plus ('ab', 'c')
	\end{lstlisting}
	
	\begin{lstlisting}
		A = []
		x = input()
		while x != '0':
			A.append(x)
			x = input()
		while A:
			print(A.pop())
	\end{lstlisting}
	
		
	\subsection*{}
	
	
	\subsection*{}
	
	
	\subsection*{}
	
	
	\subsection*{}
	
\end{document}