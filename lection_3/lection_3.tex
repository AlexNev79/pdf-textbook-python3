% !TeX spellcheck = russian_english
% !TeX encoding = UTF-8
\documentclass[a4paper, fleqn]{article}

\usepackage{indentfirst} % Красная строка
\usepackage[T2A]{fontenc} % Поддержка русских букв
\usepackage[utf8]{inputenc} % Кодировка utf8
\usepackage[russian]{babel} % руссификация
\usepackage{amssymb} % дополнительные символы
\usepackage{textcomp} % дополнительные текстовые символы
\usepackage{amsmath} % матрицы
\usepackage{listings}
% листинги
\lstset{language=Python, tabsize=4, language=Python}


\textwidth = 16 cm
\oddsidemargin= 0 cm
\evensidemargin= 1 cm


% \abovedisplayskip = 0 pt %.2\abovedisplayskip
% \belowdisplayskip = .2\belowdisplayskip
% \abovedisplayshortskip=.2\abovedisplayshortskip
% \belowdisplayshortskip=.2\belowdisplayshortskip
% \topsep= 0 pt

% \setlength{\mathindent}{1.2 cm}
% \setlength{\topsep}{0 pt}
% \setlength{\abovedisplayskip}{0 pt}

% \clubpenalty = 5000 % запрет висячих строк
% \widowpenalty = 5000
\binoppenalty=10000
\relpenalty=10000

% собственные команды и окружения

\newenvironment{example}[1][]{\medskip \noindent \textbf{Пример. #1}\par \nopagebreak}{\medskip \par} % окружение-"пример"


\title{Лекция \textnumero\,3}
% {\huge \vspace{3 cm}}}

\author{Т.\,Ф. Хирьянов}

\date{}

\begin{document}
	\maketitle
	
	\subsection*{Позиционные системы счисления}
	
	Система счисления --- это механизм кодирования чисел. Сами числа существуют независимо от формы его записи. 
	Например, тысяча в римской системе счисления записывается, как $M$, а в арабской -- $1000$. При этом эти записи обозначают одно и то же число. 
	Самой простой системой счисления является \emph{унарная}.
	В ней существует только один символ, который в случае числа $A$ ставится подряд $A$ раз.
	
	Числа записываются с помощью цифр. Количество цифр соответствует названию системы. Например, в троичной СС их 3 (0, 1, 2), в двоичной -- 2 (0, 1).
	Рассмотрим запись числа в  десятичной системе счисления 
	\[12345_{10} = 10000+2000+300+40+5 = 1*10^{4} + 2*10^3+3*10^2 + 4*10^1 + 5*10^0\]
	а также в двоичной
	\[1010101_2 = 2^6+2^4+2^2+2^0 = 64+16+4+1 = 85\]
	Стоит отметить, что в двоичной системе таблицы умножения и сложения выглядят очень просто.
	\qquad { } \\ 
	\begin{tabular}{c|cc}
		$\times$ & 0 & 1 \\
		\hline
		0 & 0 & 0 \\
		1 & 0 & 1
	\end{tabular}\qquad
	%  
	\begin{tabular}{c|cc}
		$+$ & 0 & 1 \\
		\hline
		0 & 0 & 1 \\
		1 & 1 & $10_2$
	\end{tabular}
	\\
	% СНОСКА: 12 vs 10
	Также она является  наиболее удобной для компьютера.
	
	Для того чтобы выполнить обратную операцию воспользуемся представлением числа по схеме Горнера. Например, 
	\[12345_{10}= (((1*10+2)*10+3)*10+4)*10+5\] 
	Из данного представления легко заметить, что последняя цифра числа есть остаток от деления на основание СС. Действительно, 
	\[12345~\%~10 == 5\]
	Пусть теперь 
	\[y = 12345~div~10 == 1234\]
	Тогда 
	\[y~\%~10 == 4\]
	Продолжая аналогично получим все цифры числа. Для расчетов на бумаге удобно пользоваться следующим представлением. 
	% вставка решения "в столбик" 17:25
	%\begin{tabular}{c}
		
	%\end{tabular}
	Данный алгоритм можно организовать в виде цикла. 
	Для простоты можно написать программу, выводящую на экран цифры числа в обратном порядке, причем само число меняется по ходу выполнения программы.
	\texttt{
		\begin{tabbing}
			\hspace{8mm}\=\hspace{8mm}\=\hspace{2cm}\=\kill
			x = int(input())\\
			while x != 0:\\
			\> print(x \% 10)\\
			\> x //= 10
		\end{tabbing}
	}		
	Существуют родственные двоичной системы счисления. Это четверичная, восьмеричная, шестнадцатиричная и т.д. 
	
	Рассмотрим таблицу соответствий этих четырех систем.
	% 25:30
	
	\begin{tabular}{cccc|cccc}
		<<2>> & <<4>> & <<8>> & <<16>> & <<2>> & <<4>> & <<8>> & <<16>>\\
		\hline
		0 & 0 & 0 & 0 & 1000 & 20 & 10 & 8\\ 
		1&  1&  1&  1&  1001&  21&  11&  9\\ 
		10&  2&  2&  2&  1010&  22&  12&  A\\ 
		11&  3&  3&  3&  1011&  23&  13&  B\\ 
		100&  10&  4&  4&  1100&  30&  14&  C\\ 
		101&  11&  5&  5&  1101&  31&  15&  D\\ 
		110&  12&  6&  6&  1110&  32&  16&  E\\ 
		111&  13&  7&  7&  1111&  33&  17&  F\\ 
	\end{tabular}  
	
	Внимательно посмотрев на нее, можно заметить, что когда заканчивается разряд в шестнадцатиричной системе счисления, в двоичной тоже оказываются исчерпанными уже четыре разряда.
	Аналогично разряду в восьмеричной системе счисления соответствует три разряда в двоичной, а в четвертичной --- два.
	Так, например, для перехода из шестнадцатиричной системы необходимо вместо каждой цифры подставить ее четырехразрядное представление в двоичной системе. 
	\[3DE80C_16 = 0011\,1101\,1110\,1000\,0000\,1100_2\]
	
	\subsection*{Классификация числовых алгоритмов}
	 
	\begin{enumerate}
		\item \emph{число $\rightarrow$ число}\\
		Такие алгоритмы получаются при определении значения функции от данного числа.
		\item \emph{число $\rightarrow$ последовательность}\\
		С помощью таких алгоритмов по заданным числам (некоторым параметрам) генерируется последовательность. Обычно она вычисляется либо по выражению для ее члена, либо при помощи рекуррентных соотношений.
		\item \emph{последовательность $\rightarrow$ число}\\
		В данных алгоритмах обычно реализованны различные подсчеты (например, количества элементов, их  суммы или произведения), поиск чисел с заданными свойствами, проверка упорядоченности.
		\item \emph{последовательность $\rightarrow$ последовательность}\\
		Например, фильтрация.
	\end{enumerate}
	% 34:57
	
	В третьем пункте обработка последовательности включает цикл, в котором поочередно пробегаются по ее элементам. При этом в зависимости от задачи перед переходом к следующей итерации реализуются соответствующие <<микроалгоритмы>>.
	
	Так, при подсчете количества элементов $n$, до начала  цикла полагается $n=0$, а затем в теле цикла выполняется всего одна команда $n$ += $1$.
	
	При подсчете суммы $s$  для начальной, пустой последовательности полагается $s=0$. Это, конечно, не обоснованно, однако уже после первой итерации сумма станет верной. При этом выполнится команда $s$ += $x$.
	
	То же самое касается произведения $p$. Только в этом случае вначале полагается $p=1$, чтобы на первой итерации, содержащей команду $p$ *= $x$ значение произведения не обратилось в ноль и итоговый результат не был искажен. 
	
	При поиске происходит сравнение текущего элемента с некоторым образцом. Результатом этого сравнения является логическое значение (true или false), которое может быть приведено к целому числу (соответственно 1 или 0). 
	В данном алгоритме используется специальная переменная~-- флаг~($f$). Вначале полагается $f=0$. Затем в каждой итерации выполняется $f=int(x == x_0)$. Тем самым по завершении цикла $f$ будет содержать количество элементов, равных $x_0$. 
	
	При реализации данных алгоритмов сама последовательность может быть введена двумя способами. В первом из них вначале программы может быть поданно число элементов последовательности. Тогда их считывание удобно организовать при помощи цикла \emph{for}, считывая вначале каждой итерации текущий элемент. При подсчете суммы это может выглядеть, например, таким образом:
	\texttt{
		\begin{tabbing}
			\hspace{8mm}\=\hspace{8mm}\=\hspace{2cm}\=\kill
			N = int(input())\\
			s = 0\\
			for i in range(N):\\
			\> x = float(input())\\
			\> s += x\\
			print(s)
		\end{tabbing}
	}		
		
	Когда же исходно неизвестно количество элементов, но известно, что признаком конца последовательности является, например, число $0$, удобно организовать считывание элементов иначе, используя цикл \emph{while}. При этом число $0$ не является последним элементом (в случае  подсчета произведения это  привело бы к его обнулению). В данном случае ноль --- это \emph{терминальный признак}. Переход к телу цикла осуществляется, только если выполнено условие $x != 0$, т.\,е. когда значение $x$ нетерминальное. 
	При этом необходимо считать первый элемент до начала цикла, а  последующие в конце каждой итерации.
	Тот же самый подсчет суммы может выглядеть следующим образом: 
	\texttt{
		\begin{tabbing}
			\hspace{8mm}\=\hspace{8mm}\=\hspace{2cm}\=\kill
			s = 0\\
			x = float(input())\\
			while x != 0:\\
			\> s += x\\
			\> x = float(input())\\
			print(s)	
		\end{tabbing}
	}
	В данной программе ноль будет считан в последней итерации, однако не войдет в сумму, так как при последующей проверке условия цикл завершится. 
	
	% СНОСКА
	Такой механизм используется, например, в языке Си, где признаком конца файла является специальный символ \emph{EOF}. 
	
	Фильтрация, как правило, не реализуется отдельно, используется при выполнении другой задачи, например, при подсчете среднего арифметического только четных элементов последовательности. Для этого в теле цикла команды-обработчики заключаются внутри условной конструкции, проверяющей, является ли элемент четным числом. Например, так.
	\texttt{
		\begin{tabbing}
			\hspace{8mm}\=\hspace{8mm}\=\hspace{2cm}\=\kill
			if x \% 2 == 0:\\
			s += 0\\
			n += 1\\
			print(s/n)
		\end{tabbing}
	}
	В качестве примера генерации рекурсивной последовательности можно рассмотреть выведение на экран $N$-ого числа Фибоначчи. В данной последовательности первые два элемента равны 1, а каждый последующий -- сумме двух предыдущих. 
	\[1~1~2~3~5~8~13~21 ...\]
	Из этого следует, что для корректной работы программы не требуется хранить в памяти компьютера все члены. Для то чтобы получить следующий элемент, достаточно помнить только последнии два. Вначале алгоритма вводятся две переменные $f1 = 0$, $f2 = 1$ и счетчик $n = 1$. Можно заметить, что при таких значениях элемент с номером $n$ хранится в $f2$. Это утверждение должно быть верным на протяжении всего времени работы программы. Пока номер $n$ меньше $N$, значение $f1$ меняется на $f2$, а $f2$ -- на сумму $f1$ и~ $f2$. Последнее удобно организовать с помощью присваивания кортежей. В конце итерации, конечно, необходимо увеличить счетчик $n$. 
	\texttt{
		\begin{tabbing}
			\hspace{8mm}\=\hspace{8mm}\=\hspace{2cm}\=\kill
			N = int(input())\\
			f1 = 0\\
			f2 = 1\\
			n = 1\\
			while n < N:\\
			\> f1, f2 = f2, f1 + f2\\
			\> n += 1\\
			print(f2)
		\end{tabbing}
	}	
	\subsection*{Запись чисел в Python}
	%54:46
	
	В Python существует способ записи чисел в отличной от десятичной системе счисления. 
	Например, введенное число number может быть записано в семеричной системе счисления, указав в качестве второго аргумента функции int() ее основание.
	\texttt{
		\begin{tabbing}
			\hspace{8mm}\=\hspace{8mm}\=\hspace{2cm}\=\kill
			number = input()\\
			x = int(number, 7)
		\end{tabbing}
	}
	При этом в памяти компьютера числа будут храниться всегда в двоичной системе счисления и только во время вывода они приобретут необходимый вид. 
	
	Существует также способ задать переменной значение в виде числа в двоичной, восьмеричной или шестнадцатеричной системе счисления. Для этого в начале числа необходимо записать специальный литерал: $0b$ для двоичной, $0o$ для восьмеричной и $0x$ для шестнадцатеричной. 
	\texttt{
		\begin{tabbing}
			\hspace{8mm}\=\hspace{8mm}\=\hspace{2cm}\=\kill
			x = 0b10110\\
			x = 0o756\\
			x = 0xB708
		\end{tabbing}
	}
	\subsection*{Однопроходные алгоритмы}
	
	В приведенных ранее алгоритмах количество операций линейно зависело от размера последовательности чисел, а объем требуемой памяти был конечным и не зависел этого размера. 
	
	Бывает, что в сложных задачах математика может помочь уменьшить количество вычислений. 
	
	Так, например, вычисление среднего квадратичного отклонения от средней величины на первый взгляд осуществляется в два прохода, и в памяти требуется хранить все числа.
	\[\sigma = \sqrt{\dfrac{\sum_{i=1}^{N}(x_i - \bar{x})^2)}{N}}\]
	Действительно, кажется, что сначала нужно подсчитать $\bar{x}$, а затем уже само отклонение. Однако если раскрыть квадрат и произвести суммирование, то можно получить известную формулу.
	\[\sigma^2 = \dfrac{\sum_{i=1}^{N}({x_i}^2 - 2x_i\bar{x} + \bar{x_i}^2)}{N}=\]
	\[=\dfrac{\sum_{i=1}^{N}{x_i}^2}{N} - 2\bar{x}^2 + \dfrac{\bar{x}^2}{N}N\]
	Откуда
	\[\sigma = \sqrt{\bar{x^2} - \bar{x}^2}\]
	Из этой формулы следует, что вычислить отклонение можно за один проход, определяя среднее значение $x$ и его средний квадрат. 
	
	\subsection*{Логические операции}
	
	В питоне существуют логические операции \emph{and}, \emph{or} и \emph{not}. Они возвращают значения \emph{True} или \emph{False} (в качестве обозначений будут использоваться соответственно 1 и 0) и имеют следующие таблицы истинности.
	
	\begin{tabular}{cc|c|cc|c}
		A&  B&  A and B&  A&  B&  A or B\\ 
		\hline
		0&  0&  0&  0&  0&  0\\ 
		0&  1&  0&  0&  1&  1\\ 
		1&  0&  0&  1&  0&  1\\ 
		1&  1&  1&  1&  1&  1\\ 
	\end{tabular} 
	%
	\begin{tabular}{c|c}
		A&  not A\\ 
		\hline  
		0&  1\\ 
		1&  0\\  
	\end{tabular} 
\end{document}
