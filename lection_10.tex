% !TeX spellcheck = russian_english
% !TeX encoding = UTF-8
\documentclass[a4paper, fleqn]{article}

\usepackage{indentfirst} % Красная строка
\usepackage[T2A]{fontenc} % Поддержка русских букв
\usepackage[utf8]{inputenc} % Кодировка utf8
\usepackage[russian]{babel} % руссификация
\usepackage{amssymb} % дополнительные символы
\usepackage{textcomp} % дополнительные текстовые символы
\usepackage{amsmath} % матрицы
%\usepackage{pdfpages}
%\usepackage{pgfplots}
%\usepackage{pgfplotstable}
\usepackage{listings}
% листинги
\lstset{language=Python, tabsize=4, language=Python}


\textwidth = 16 cm
\oddsidemargin= 0 cm
\evensidemargin= 1 cm


% \abovedisplayskip = 0 pt %.2\abovedisplayskip
% \belowdisplayskip = .2\belowdisplayskip
% \abovedisplayshortskip=.2\abovedisplayshortskip
% \belowdisplayshortskip=.2\belowdisplayshortskip
% \topsep= 0 pt

% \setlength{\mathindent}{1.2 cm}
% \setlength{\topsep}{0 pt}
% \setlength{\abovedisplayskip}{0 pt}

% \clubpenalty = 5000 % запрет висячих строк
% \widowpenalty = 5000
\binoppenalty=10000
\relpenalty=10000

% собственные команды и окружения

\newenvironment{example}[1][]{\medskip \noindent \textbf{Пример. #1}\par \nopagebreak}{\medskip \par} % окружение-"пример"


\title{Лекция \textnumero\,10\\
	Пока только листинги кода}
% {\huge \vspace{3 cm}}}

\author{Т.\,Ф. Хирьянов}

\date{}

\begin{document}
	\maketitle
	
\section*{Динамическое программирование}

% мой листинг
\texttt{
	\begin{tabbing}
		\hspace{8mm}\=\hspace{8mm}\=\hspace{8mm}\=\hspace{3cm}\=\kill
		р
	\end{tabbing}
	}

%04:00


\texttt{
	\begin{tabbing}
		\hspace{8mm}\=\hspace{8mm}\=\hspace{8mm}\=\hspace{3cm}\=\kill
		def fib(n):\\
		\> F = [0, 1] + [0] + (n - 1)\\
		\> for i in range(2, n + 1):\\
		\> \> F[i] = F[i - 1] + F[i - 2]\\
		\> return F[n]
	\end{tabbing}
}

% 10:15

% 14:40
\texttt{
	\begin{tabbing}
		\hspace{8mm}\=\hspace{8mm}\=\hspace{8mm}\=\hspace{3cm}\=\kill
		K = [0, 1] + [0] + (n - 1)\\
		for i in range(2, n + 1):\\
		\> \> K[i] = K[i - 1] + K[i - 2]
	\end{tabbing}
}

% 16:30
\subsection*{Количество траекторий с запрещенными клетками}

\texttt{
	\begin{tabbing}
		\hspace{8mm}\=\hspace{8mm}\=\hspace{8mm}\=\hspace{3cm}\=\kill
		Denied = [0, 0, 1, 0, 0, 1, 0, 0, ... 1, 0, 0]\\
		n = int(input())\\
		K =[0, 1] + [None]*(n - 1)\\
		for i in range(2, n + 1):\\
		\> if Denied[i]:\\
		\> \> K[i] = 0\\
		\> else:\\
		\> \> K[i] = K[i - 1] + K[i - 2]
	\end{tabbing}
}

% 22:50
\subsection*{Задача о минимальной стоимости}

% 29:30
\texttt{
	\begin{tabbing}
		\hspace{8mm}\=\hspace{8mm}\=\hspace{8mm}\=\hspace{3cm}\=\kill
		Price = [10, 20, 5, 3, ...]\\
		n = int(input())\\
		C = [0]*(n + 1)\\
		C[0] = Price[0]\\
		C[1] = C[0] + Price[1]\\
		for i in range(2, n + 1):\\
		\> C[i] = Price[i] + min(C[i - 1], C[i - 2])\\
		print(C[n])\\
		\\
		Path = [n]\\
		while Path[-1] != 0:\\
		\> i = Path[-1]\\
		\> if C[i - 1] < C[i - 2]:\\
		\> \> Path.append(i - 1)\\
		\> else:\\
		\> \> Path.append(i - 2)\\
		Path = Path[::-1]
	\end{tabbing}
}

% перерыв

% король
% 00:00

% 03:30

\subsection*{Исполнитель король}
\texttt{
	\begin{tabbing}
		\hspace{8mm}\=\hspace{8mm}\=\hspace{8mm}\=\hspace{3cm}\=\kill
		m = int(input())\\
		r = int(input())\\
		K = [[0]*m for i in range(n)]\\
		for i in range(n):\\
		\> K[i][0] = 1\\
		for j in range(m):\\
		\> K[0][j] = 1\\
		for i in range(1, n):\\
		\> for j in range(1, m):\\
		\> \> K[i][j] = K[i - 1][j - 1] + K[i - 1][j] + K[i][j - 1]\\
		print(K[n - 1][m - 1])
	\end{tabbing}
}

% 15:00
\subsection*{Наибольшая общая подпоследовательность}

% 16:15 рисунок 2 последовательности

% 24:00
\texttt{
	\begin{tabbing}
		\hspace{8mm}\=\hspace{8mm}\=\hspace{8mm}\=\hspace{3cm}\=\kill
		n = len(A)\\
		m = len(B)\\
		F = [[0]*(m + 1) for i in range(n + 1)]\\
		for i in range(1, n + 1):\\
		\> for j in range(1, m + 1):\\
		\> \> if A[i - 1] == B[j - 1]: \> \> \# i символ А \\
		\> \> \> F[i][j] = F[i - 1][j - 1] + 1\\
		\> \> else:\\
		\> \> \> F[i][j] = max(F[i - 1][j], F[i][j - 1])\\
		print(F[n][m])
	\end{tabbing}
}

\subsection*{Наибольшая возрастающая подпоследовательность}

% 32:00

% 35:00
\texttt{
	\begin{tabbing}
		\hspace{8mm}\=\hspace{8mm}\=\hspace{8mm}\=\hspace{3cm}\=\kill
		F = [0] + len(A)\\
		for i in range(len(A)):\\
		\> for j in range(i):\\
		\> \> if A[j] < A[i] $\backslash$\\
		\> \> \> and F[j] > F[i]:\\
		\> \> \> F[i] = F[j]\\
		\> F[i] += 1\\
		print(max(F))
	\end{tabbing}
}

\end{document}