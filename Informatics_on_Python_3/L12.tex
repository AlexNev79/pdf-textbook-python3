\chapter{}
\subsection*{Объектно-ориентированное программирование}

Объектно-ориентированное программирование держится на трех китах. Это
\begin{enumerate}
	\item Инкапсуляция
	\item Полиморфизм 
	\item Наследование
\end{enumerate}

Модульная парадигма является частью инкапсуляции и заключается в объединении в группы функций с некоторым общим назначением (например, в один модуль удобно собрать функции, задействованные в вводе и интерпретации данных). Такая парадигма широко распространена в не объектно-ориентированных языках программирования. 

Следующим шагом к инкапсуляции является введение \emph{классов} Это структуры (типы) данных, которые обладают определенными \emph{атрибутами}, а также \emph{методами} -- функциями с помощью которых осуществляется доступ к атрибутам.

Возможно существование нескольких экземпляров класса -- объектов. Так могут существовать два объекта: заяц Вася и заяц Дима. Оба они будут являться экземплярами класса заяц (Rabbit). Заяц имеет два  свойства (атрибута): размер (size) и сытость (sytost). К ним можно обратиться с помощью методов класса: set\_size, устанавливающего определенный размер size, get\_size, возвращающего значение size, и  feed, увеличивающего сытость. Все эти функции имеют обязательный аргумент self, стоящий на первом месте. Он является ссылкой на экземпляр класса, и должен быть обязательно указан в описании метода (только так метод <<поймет>>, для которого экземпляра класса он вызван).
\texttt{
	\begin{tabbing}
		\hspace{8mm}\=\hspace{8mm}\=\hspace{8mm}\=\hspace{3cm}\=\kill
		class Rabbit:\\
		\> size = None\\
		\> sytost = 0\\
		\> def set\_size(self, size):\\
		\> \> self.size = size\\
		\> def get\_size(self):\\
		\> \> return self.size\\
		\> def feed(self, meal):\\
		\> \> self.sytost += calorii(meal)
	\end{tabbing}
}

Стоит отметить, что обращаться к атрибутам и манипулировать ими удобнее не на прямую, а с помощью методов класса. Тогда можно производить действия с атрибутами, используя понятный интерфейс, не задумываясь, как именно они осуществляются. Из приведенного ниже примера ясно, что кролика a покормили морковкой. При этом мы не знаем, что это подразумевает под собой.
\texttt{
	\begin{tabbing}
		\hspace{8mm}\=\=\hspace{3cm}\=\kill
		a = Rabbit()\\
		b = Rabbit()\\
		a.feed(carrot) \> \> \> \# Rabbit.feed(a, carrot)
	\end{tabbing}
}

\subsection*{Полиморфизм в Python}

В питоне невозможно перегружать функции. В рассмотренном ниже примере происходит не перегрузка, а переопределение функции f -- старое определение пропадает. 
\texttt{
	\begin{tabbing}
		\hspace{8mm}\=\hspace{8mm}\=\hspace{8mm}\=\hspace{3cm}\=\kill
		def f(x):\\
		\> return x*x\\
		def f(x, y):\\
		\> return x*y
	\end{tabbing}
}
Для того чтобы обойти это препятствие можно воспользоваться значениями по умолчанию. Допустим, что функция f должна возвращать произведение двух поданных на вход чисел и квадрат числа, если подано лишь одно число. Для обработки последней ситуации удобно определить по умолчанию один из аргументов значением None. 
\texttt{
	\begin{tabbing}
		\hspace{8mm}\=\hspace{8mm}\=\hspace{8mm}\=\hspace{3cm}\=\kill
		def f(x, y=None):\\
		\> if y == None:\\
		\> \> y = x\\
		\> return x*y
	\end{tabbing}
}
Можно однако использовать аргументы разного типа. И внутри функции определить, как вести себя с теми или иными объектами.  Этим и достигается полиморфизм в Python.

\subsection*{setter и getter}

Создадим класс студента. У него необходимо создать атрибуты возраст и имя, причем они должны определяться автоматически при создании объекта класса student. Это делается с помощью специального метода-конструктора \_\_init\_\_. Для того чтобы сделать доступ к атрибуту извне нежелательным, его имя должно начинаться с символа <<\_>> (тогда данного атрибута не будет видно в списке атрибутов класса). Доступ к атрибуту обычно осуществляется с помощью функции <<getter>>, которая имеет такое же название, что и атрибута и перед ее определением написано \texttt{@property}. Для того чтобы установить желаемое значение аргумента, правильнее использовать setter. Его определение аналогично.
\texttt{
	\begin{tabbing}
		\hspace{8mm}\=\hspace{8mm}\=\hspace{8mm}\=\hspace{4.5cm}\=\kill
		class Student:\\
		\> def \_\_init\_\_(self, age, name):\> \> \> \# Конструктор\\
		\> \> self.\_age = age\\
		\> \> self.\_name = name\\
		\> def aging(self):\\
		\> \> self.\_age += 1\\
		\> \> print('Ура! Мне', self.\_age, 'лет!')\\
		\> @property\\
		\> def age(self): \> \> \> \# <<getter>> (свойство)\\
		\> return self.\_age\\
		\> @age.setter\\
		\> def age(self, age):\\
		\> \> self.age = age
	\end{tabbing}
}
Тогда можно возвратить значение атрибута показанным ниже способом (здесь осуществляется не прямой доступ к атрибуту, а с помощью функции getter)
\texttt{
	\begin{tabbing}
		\hspace{8mm}\=\hspace{8mm}\=\hspace{8mm}\=\hspace{3cm}\=\kill
		a = Student(17, 'Вася')\\
		a.age = 18 \> \> \> \# Нельзя!\\
		x = a.age
	\end{tabbing}
}

\subsection*{Определение функций}
Важно отметить, что к объектам класса можно приписывать новые атрибуты уже после определения класса. 
Так как любая функция является объектом класса функций, то и для нее можно создать атрибуты. Например, атрибут p функции productor можно задать уже после объявления функции. Причем она будет работать только после этого.
\texttt{
	\begin{tabbing}
		\hspace{8mm}\=\hspace{8mm}\=\hspace{8mm}\=\hspace{2cm}\=\kill
		def productor(x):\\
		\> return x*productor.p\\
		productor.p = 5\\
		print(productor(2)) \> \> \> \> \# 10
	\end{tabbing}
}
Рассмотрим класс двумерного вектора Vector2D. Допустим мы хотим сложить два вектора не с помощью некоторой функции, а привычным способом (c = a + b). Для этого нужно определить специальную функцию \_\_add\_\_. Таким же образом можно организовать умножение (\_\_mul\_\_). Необходимо однако различать скалярное произведение и умножение вектора на число. Это делается с помощью проверки принадлежности (isInstance) объекта тому или иному типу (или его прямому потомку).
\texttt{
	\begin{tabbing}
		\hspace{8mm}\=\hspace{8mm}\=\hspace{8mm}\=\hspace{3cm}\=\kill
		class Vector2D:\\
		\> def \_\_init\_\_(self, x=0, y=0):\\
		\> \> self.\_x = x\\
		\> \> self.\_y = y\\
		\> def \_\_add\_\_(self, other):\\
		\> \> x = self.\_x + other.\_x\\
		\> \> y = self.\_y + other.\_y\\
		\> \> return Vector2D(x, y)\\
		\> def \_\_mul\_\_(self, other):\\
		\> \> if isInstance(other, Vector2D):\\
		\> \> \> return self.\_x *other.\_x + self.\_y *other.\_y\\
		\> \> else:\\
		\> \> \> return Vector2D(self.\_x * other, self.\_y * other)
	\end{tabbing}
}
\subsection*{range}
Допустим, что нужно создать итератор, который бы не сразу создавал весь список, а возвращал бы один элемент и останавливался на паузу. Например, нужно генерировать геометрическую прогрессию. Это можно осуществить следующим образом. Приведенная ниже функция является функцией-генератором. Вместо return используется yield.
\texttt{
	\begin{tabbing}
		\hspace{8mm}\=\hspace{8mm}\=\hspace{8mm}\=\hspace{3cm}\=\kill
		def geom\_range(start, stop, d):\\
		\> x = start\\
		\> while x < stop:\\
		\> \> yield x\\
		\> \> x *= d
	\end{tabbing}
}
Затем данную функцию можно использовать как итератор.
\texttt{
	\begin{tabbing}
		\hspace{8mm}\=\hspace{8mm}\=\hspace{8mm}\=\hspace{3cm}\=\kill
		for x in geom\_range(1, 1025, 2):\\
		\> print(x)
	\end{tabbing}
}
Многие функции умеют работать с данными итераторами.
\texttt{
	\begin{tabbing}
		\hspace{8mm}\=\hspace{8mm}\=\hspace{8mm}\=\hspace{3cm}\=\kill
		g\_num = dict(zip(range(0, 11), geom\_range(1, 1025, 2))\\
		x, y, z = list(map(int, input().split())\\
		x, y, z = [int(x) for x in input().split()]
	\end{tabbing}
}
\subsection*{Списки level up}
Очевидно, что в списках могут храниться другие списки. Однако в действительности там хранятся ссылки на объекты. В приведенном ниже примере последний элемент списка и вовсе хранит ссылку на сам список. 
\texttt{
	\begin{tabbing}
		\hspace{8mm}\=\hspace{8mm}\=\hspace{8mm}\=\hspace{3cm}\=\kill
		a = [1]\\
		a.append(a) \> \> \> \# [1, *]\\
		x = a\\
		for i in range(10):\\
		\> print(x[0])\\
		\> x = x[1]\\
	\end{tabbing}
}
Данный алгоритм может быть основой для структуры данных, представляющей собой систему вложенных списков, таких что последний элемент n - 1 списка содержит ссылку на n список. Это \emph{односвязный} список. Его преимущество состоит в том, что сложность включения нового элемента, например, в середину списка составляет O(1), в отличие от О(N) для массива.
